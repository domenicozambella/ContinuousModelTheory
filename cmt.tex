\documentclass[11pt,oneside]{amsproc}

\usepackage[utf8]{inputenc}
\usepackage[english]{babel}
\usepackage{comment}
\usepackage{amsrefs}
\usepackage{tikz}
\usepackage{xcolor}
\usepackage{datetime2} 
\usepackage[colorlinks=true,linkcolor=blue,]{hyperref}

\usepackage[a4paper,margin=30mm]{geometry}
\usepackage{stackengine}
\usepackage{scalerel}
\usepackage{mathtools}
\usepackage{calc}
\usepackage{amsthm}
\usepackage{thmtools}
\usepackage[framemethod=TikZ]{mdframed}
\usepackage{amssymb}
\usepackage{amsfonts}
\usepackage[mathcal]{euscript}
% \usepackage{fourier-orns}
% \usepackage{palatino}
\usepackage{mathpazo} % add possibly `sc` and `osf` options
%\usepackage{eulervm}
%\usepackage{bbm}
%\usepackage{latexsym}
% \usepackage{mathrsfs}
%\usepackage{stmaryrd}
\usepackage{stix}
\usepackage{dsfont}
% \newcommand*{\TakeFourierOrnament}[1]{{%
% \fontencoding{U}\fontfamily{futs}\selectfont\char#1}}
% \renewcommand*{\danger}{\TakeFourierOrnament{66}}
\parindent0ex
\parskip1.2ex
\makeatletter
\DeclareOldFontCommand{\rm}{\normalfont\rmfamily}{\mathrm}
\DeclareOldFontCommand{\sf}{\normalfont\sffamily}{\mathsf}
\DeclareOldFontCommand{\tt}{\normalfont\ttfamily}{\mathtt}
\DeclareOldFontCommand{\bf}{\normalfont\bfseries}{\mathbf}
\DeclareOldFontCommand{\it}{\normalfont\itshape}{\mathit}
\DeclareOldFontCommand{\sl}{\normalfont\slshape}{\@nomath\sl}
\DeclareOldFontCommand{\sc}{\normalfont\scshape}{\@nomath\sc}
\makeatother

\newcommand{\mylabel}[1]{{#1}\hfill}
\renewenvironment{itemize}
  {\begin{list}{$\cdot$}{%
  \setlength{\parskip}{0mm}
  \setlength{\topsep}{.2\baselineskip}
  \setlength{\rightmargin}{0mm}
  \setlength{\listparindent}{0mm}
  \setlength{\itemindent}{0mm}
  \setlength{\labelwidth}{3ex}
  \setlength{\itemsep}{.2\baselineskip}
  \setlength{\parsep}{.2\baselineskip}
  \setlength{\partopsep}{0mm}
  \setlength{\labelsep}{1ex}
  \setlength{\leftmargin}{\labelwidth+\labelsep}
  \let\makelabel\mylabel}}{%
\end{list}}

\declaretheoremstyle[
  headfont=\normalfont\bfseries,
  notefont=\bfseries,
  notebraces={(}{)},
  bodyfont=\normalfont,
  postheadspace=1em,
  mdframed={
  outerlinewidth=1pt,
  linecolor=gray!25,
  roundcorner = 1ex,    
  backgroundcolor=gray!15, 
  innerleftmargin=1ex,
  leftmargin=-1ex,
  innerrightmargin=1ex,
  rightmargin=-1ex,
  innertopmargin=1.5ex, 
  innerbottommargin=1ex, 
  skipabove=3ex,
  skipbelow=1ex}, 
]{mystyle}

\declaretheorem[style=mystyle]%
{theorem}
\declaretheorem[style=mystyle,sibling=theorem]%
{lemma,proposition,fact,corollary,definition,remark,example,claim,question}

\let\proof\relax
\declaretheoremstyle[
  spaceabove=6pt, 
  spacebelow=6pt, 
  headfont=\normalfont\itshape, 
  bodyfont = \normalfont,
  postheadspace=1em, 
  qed=\qedsymbol, 
  headpunct={.}]
{myproof} 
\declaretheorem[style=myproof, unnumbered]{proof}

\renewcommand*{\emph}[1]{%
   \smash{\tikz[baseline]\node[rectangle, fill=teal!25, rounded corners, inner xsep=0.5ex, inner ysep=0.2ex, anchor=base, minimum height = 2.7ex]{#1};}}

\linespread{1.1}
\author{C. L. C. L. Polymath}
\begin{document}
\title{Continuous logic for the classical logician}
\hfill\texttt{Branch:\ main\ \DTMnow}\bigskip
\maketitle
\raggedbottom

\def\forallM{\forall\raisebox{1.1ex}{\scaleto{\sf M}{.8ex}\kern-.2ex}}
\def\existsM{\exists\raisebox{1.1ex}{\scaleto{\sf M}{.8ex}\kern-.2ex}}
\def\forallR{\forall\raisebox{1.1ex}{\scaleto{\sf R}{.8ex}\kern-.2ex}}
\def\existsR{\exists\raisebox{1.1ex}{\scaleto{\sf R}{.8ex}\kern-.2ex}}

\newcommand\questionsign[1][2ex]{%
  \renewcommand\stacktype{L}%
  \scaleto{\stackon[-.6pt]{\color{red}$\triangle$}{\color{red}\bfseries\small ?}}{#1}%
}

\newcommand\dangersign[1][2ex]{%
  \renewcommand\stacktype{L}%
  \scaleto{\stackon[-.6pt]{\color{red}$\triangle$}{\color{red}\bfseries\small !}}{#1}%
}


%%%%%%%%%%%%%%%%%%%%%%%%%%%%%%%%%%%
%%%%%%%%%%%%%%%%%%%%%%%%%%%%%%%%%%%
%%%%%%%%%%%%%%%%%%%%%%%%%%%%%%%%%%%
%%%%%%%%%%%%%%%%%%%%%%%%%%%%%%%%%%%
%%%%%%%%%%%%%%%%%%%%%%%%%%%%%%%%%%%
\section{Introduction}\label{intro}


\def\ceq#1#2#3{\parbox[t]{23ex}{$\displaystyle #1$}\parbox{6ex}{\hfil $#2$}{$\displaystyle #3$}}

As a minimal motivating example, consider a real vector space $M$.
This is are among the most simple strucures considered in model theory.
Now, expand it by adding a norm.
A norm is a function that, given vector, outputs a real number.
We may formalize this in a natural way by using a two-sorted structure $\langle M,\mathds{R}\rangle$.
Now, we have two conflicting aspirations
\begin{itemize}
  \item[i.] to apply the basic tools of model theory (such as elementarity, compactness, and saturation);
  \item[ii.] to stay within the realm of normed spaces (hence insist that 
  $\mathds{R}$ should remain $\mathds{R}$ throughout).
\end{itemize}

These two requests are blatantly incompatible, something has to give.
Different people have attemped different paths.
\begin{itemize}
  \item[1.] Nonstandard analysts happily embrace norms that take values in some ${}^*\mathds{R}\succeq\mathds{R}$.
  These normed spaces are nonstandard, but there are tricks to transfer results from nonstandard to standard normed spaces and vice versa.
  Model theorists, when confronted with similar problems, have used similar approaches (in more generality and with different notation).
  \item[2.] Henson and Iovino propose to stick to the standard notion of norm but restrict the notion of elementarity and saturation to a smaller class of formulas (which here is denoted by $\mathds{H}$).
  They demostrate that many standard tools of model theory apply in their setting. 
  The work of Henson and Iovino is focussed on Banach spaces and has hardly been applied to broader contexts. See~\cite{HI} for a survey.
  \item[3.] Real valued logicians take the most radical approach.
  They abandon classical true/false valued logic in favour of a logic valued in the interval $[0,1]$.
  Unfortunately, this adds notational burden and restricts the class of structures that can be considered. See~\cite{K} for a survey. 
\end{itemize}

We elaborate on the ideas of Henson and Iovino and generalize them to a larger class of structures.
Following Henson and Iovino, we restrict the notions of elementarity and saturation.
But we use a class of formulas $\mathds{I}$ which is larger than $\mathds{H}$.
Though $\mathds{I}$ has approximatively the same expressive power as $\mathds{H}$ (cf.\@ Propositions~\ref{prop_LHapprox1} and~\ref{prop_LHapprox2}), it is evident (cf.\@ Example~\ref{ex_Rvlogic}) that $\mathds{I}$ is more convinent to use.

We also borrow intuitions and results from nonstandard analisys to an extent that we could claim the title: nonstandard analysis for the finite logician.


%%%%%%%%%%%%%%%%%%%%%%%%%%%%%%%%%%%
%%%%%%%%%%%%%%%%%%%%%%%%%%%%%%%%%%%
%%%%%%%%%%%%%%%%%%%%%%%%%%%%%%%%%%%
%%%%%%%%%%%%%%%%%%%%%%%%%%%%%%%%%%%
%%%%%%%%%%%%%%%%%%%%%%%%%%%%%%%%%%%
\section{A class of structures}\label{uno}


Let \emph{$R$\/} be some fixed first-order structure which is endowed with a Hausdorff compact topology (in particular, a normal topology).
The language \emph{$L_{\sf R}$\/} contains relation symbols for the compact subsets $C\subseteq R^n$ and a function symbol for each continuous functions $f:R^n\to R$.
According to the context, $C$ and $f$ denote either the symbols of $L_{\sf R}$ or their interpretation in $R$.
% Our choice is a compromise to avoid eccessive complications.
% It would be interesting to drop the restriction on the sort of the function symbols.


\begin{definition}\label{def_0}
  Let \emph{$L_{\sf M}$\/} be a one-sorted (first-order) language.
  By \emph{model\/} we mean an ${\EuScript L}$-structure of the form ${\EuScript M}=\langle  M,R\rangle$, where $M$ ranges over $L_{\sf M}$-structures, while $R$ is the structure above.
  We use ${\sf M}$ and ${\sf R}$ to denote the two sorts of ${\EuScript L}$.
  % The two sorts are denoted by ${\sf M}$ and ${\sf R}$.
  The language \emph{${\EuScript L}$\/} is an expansion of both $L_{\sf M}$ and $L_{\sf R}$.
  The new symbols allowed in ${\EuScript L}$ are function symbols of sort ${\sf M}^n\to {\sf R}$.
\end{definition}

In some examples we need to replace the sort ${\sf M}$ with many sorts ${\sf M}_1\dots,{\sf M}_k$.
In this case the language $L_{\sf M}=L_{{\sf M}_1\dots,{\sf M}_k}$ governs the structure $\langle M_1\dots,M_k\rangle$.
Models have the form $\langle M _1\dots,M_k, R\rangle$.
The new symbols in the language ${\EuScript L}$ are of sort ${\sf M}_1^{n_1}\dots,{\sf M}_k^{n_k}\to {\sf R}$.

The notion of model will be (inessentially) restricted once the monster model is introduced, in Section~\ref{monster}.

Clearly, saturated ${\EuScript L}$-structures exist but, with the exception of trivial cases (i.e.\@ when $R$ is finite), they are not models.
As a remedy, below we carve out a set of formulas $L\subseteq\mathds{I}\subseteq {\EuScript L}$, such that every model has an $\mathds{I}$-elementary, $\mathds{I}$-saturated extension that is a model.

Warning: as usual $L_{\sf M}$ and ${\EuScript L}$ denote both first-order languages and the corresponding set of formulas.
The symbols $\mathds{I}$ and $\mathds{H}$ defined below denote only sets of ${\EuScript L}$-formulas.

\begin{definition}\label{def_LL}
  Formulas in \emph{$\mathds{I}$\/} are defined inductively from two sorts of \emph{$\mathds{I}$-atomic\/} formulas
  \begin{itemize}
  \item[i.] formulas of the form $t(x;y)\in C$, where $C\subseteq R^n$ is compact (in the product topology) and $t(x,y)$ is a $n$-tuple of terms of sort ${\sf M}^{|x|}\times {\sf R}^{|y|}\to {\sf R}$; 
  \item[ii.] all formulas in $L_{\sf M}$.
  \end{itemize}
  We require that $\mathds{I}$ is closed under the Boolean connectives $\wedge$, $\vee$; the quantifiers $\forallM$, $\existsM $ of sort ${\sf M}$; and the quantifiers of sort ${\sf R}$ which are denoted by $\forallR $, $\existsR$.

  We write \emph{$\mathds{H}$ \/} for the set of formulas in $\mathds{I}$ without quantifiers of sort ${\sf R}$.
\end{definition}

% \noindent\llap{\textcolor{red}{\Large\dangersign}\kern1ex}\ignorespaces
Formulas in $\mathds{H}$ are a generalization of the positive bounded formulas of Henson and Iovino.
Here we also introduce the larger class $\mathds{I}$ because it offers some advantages. 
For instance, it is easy to see (cf.\@ Example~\ref{ex_Rvlogic} for a hint) that $\mathds{I}$ has at least the same expressive power as real valued logic (in fact, it is way more expressive%
\footnote{This unsubstatited claim should be interpreted as a concjecture.
A painstaking technical comparison with real valued logic is not in the scope of this indroductory exploratory paper.}).
Somewhat surprisingly, we will see (cf.\@ Propositions~\ref{prop_LHapprox1} and~\ref{prop_LHapprox2}) that the formulas in $\mathds{I}$ can be approximated by formulas in $\mathds{H}$.

\begin{example}\label{ex_Rvlogic}
  Let $R=[0,1]$, the unit interval in $\mathds{R}$.
  Let $t(x)$ be a term of sort ${\sf M}^{|x|}\to {\sf R}$.
  Then there is a formula in $\mathds{I}$ that says $\sup_{x} t(x)=\tau$.
  Indeed, consider the formula

  \ceq{\hfill\forallM x\ \big[t(x)\dotminus\tau\in\{0\}\big]}
  {\wedge}{\forallR \varepsilon \Big[\varepsilon\in\{0\}\ \vee\ \existsM x\ \big[\tau\dotminus (t(x)+\varepsilon)\in\{0\}\big]\Big]}

  which, in a more legible form, becomes

  \ceq{\hfill\forallM x\ \big[t(x)\le\tau\big]}{\wedge}{\forallR \varepsilon>0\ \existsM x\ \big[\tau\le t(x)+ \varepsilon\big].}
\end{example}

We conclude this introduction with a question.
We do not know under what conditions it is possible to extends the theory exposed below to a larger ${\EuScript L}$.
Some natural examples would require ${\EuScript L}$ to have function symbols of sort ${\sf M}^n\times {\sf R}^m\to {\sf R}$ with $m,n>0$.
Also symbols of sort ${\sf M}^n\times {\sf R}^m\to {\sf M}$ have natural applications but are hard to control.

%%%%%%%%%%%%%%%%%%%%%%%%%%%%%%%%%%%%
%%%%%%%%%%%%%%%%%%%%%%%%%%%%%%%%%%%%
%%%%%%%%%%%%%%%%%%%%%%%%%%%%%%%%%%%%
%%%%%%%%%%%%%%%%%%%%%%%%%%%%%%%%%%%%
%%%%%%%%%%%%%%%%%%%%%%%%%%%%%%%%%%%%
\section{The standard part}\label{standard_part}

In this section we recall the notion of the sandard part of an element in an elementary extension of a compact Hausdorff topological space.
Our goal is to prove Lemma~\ref{lem_st} which in turn is required for the proof of the Compactness Theorem (the reader that is willing to accept the Compactness Theorem without proof may skip this section).


Let $R\preceq{}^*\!R$.
For each $\beta\in R$ we define the type

\ceq{\hfill{\rm m}_\beta(x)}{=}{\{x\in D\ :\  D \textrm{ compact neighborhood of }\beta\}.}

The set of the realizations of ${\rm m}_\beta(x)$ in ${}^*\!R$ is known to nonstandard analysts as the monad of $\beta$. 
% We write \emph{$\alpha\approx\alpha'$\/} if $\alpha\approx\beta\approx\alpha'$ for some $\beta\in R$.
% From the following fact it follows that $\,\approx\,$ is an equivalence relation on ${}^*\!R$.
The following fact is well-known.

\begin{fact}\label{fact_uniqueness_st}
  For every $\alpha\in{}^*\!R$ there is a unique $\beta\in R$ such that $\alpha\models{\rm m}_\beta(x)$.
\end{fact}

\begin{proof}
  Negate the existence of $\beta$.
  For every $\gamma\in R$ pick some compact neighborhood $D_\gamma$ of $\gamma$, such that ${}^*\!R\ \models\ \alpha\notin D_\gamma$.
  By compactness there is some finite $\Gamma\subseteq R$ such that $D_\gamma$, with $\gamma\in\Gamma$, cover $R$.
  By elementarity these $D_{\gamma}$ also cover ${}^*\!R$.
  A contradiction.
  The uniqueness of $\beta$ follows from normality.
\end{proof}

We denote by \emph{${\rm st}(\alpha)$\/} the unique $\beta\in R$ such that $\alpha\models{\rm m}_\beta(x)$.
We write \emph{$\alpha\approx\alpha'$\/} if ${\rm st}(\alpha)={\rm st}(\alpha')$.

\begin{fact}\label{fact_st1}
  For every $\alpha\in{}^*\! R$ and every compact $C\subseteq R$

\ceq{\hfill{}^*\!R}{\models}{\alpha\in C\ \rightarrow\ {\rm st}(\alpha)\in C.}
\end{fact}

\begin{proof}
  Assume ${\rm st}(\alpha)\notin C$.
  By normality there is a compact set $D$ disjoint from $C$ that is a neighborhood of ${\rm st}(\alpha)$.
  Then  ${}^*\!R\models\alpha\in D\subseteq\neg C$.
\end{proof}

\begin{fact}\label{fact_terms_st}
  For every $\alpha\in({}^*\! R)^{|x|}$ and every function symbol $f$ of sort ${\sf R}^{|x|}\to{\sf R}$

  \ceq{\hfill{}^*\!R}{\models}{{\rm st}\big(f(\alpha)\big)=f\big({\rm st}(\alpha)\big).}
\end{fact}

\begin{proof}
  By Fact~\ref{fact_uniqueness_st} and the definition of st(-) it suffices to prove that ${}^*\!R\models f(\alpha)\in D$ for every compact neighborhood $D$ of $f\big({\rm st}(\alpha)\big)$.
  
  Fix one such $D$.
  Then ${\rm st}(\alpha)\in f^{-1}[D]$ .
  By continuity $f^{-1}[D]$ is a compact neighborhood of ${\rm st}(\alpha)$.
  Therefore ${}^*\!R\models \alpha\in f^{-1}[D]$ and, as $R\preceq{}^*\!R$ we obtain ${}^*\!R\models f(\alpha)\in D$.
\end{proof}

Let ${}^*{\!\EuScript M}=\langle M,{}^*\!R\rangle$ be an ${\EuScript L}$-structure such that $R\preceq{}^*\!R$.
The \emph{standard part of ${}^*{\!\EuScript M}$\/} is the model ${\EuScript M}=\langle M,R\rangle$ that interprets the symbols $f$ of sort ${\sf M}^n\to {\sf R}$ as the functions

\ceq{\hfill f^{\EuScript M}(a)}{=}{{\rm st}\big(f^{{}^*\!\EuScript M}(a)\big)}\hfill for all $a\in M^n$.

Symbols in $L_{\sf M}$ maintain the same interpretation.

\begin{fact}\label{fact_st2}
  Let ${}^*{\!\EuScript M}$ and ${\EuScript M}$ be as above.
  Let $t(x\,;y)$ be a term of sort ${\sf M}^{|x|}\times{\sf R}^{|y|}\to {\sf R}$.
  Then for every $a\in M^{|x|}$ and $\alpha\in({}^*\!R)^{|y|}$

  \ceq{\hfill t^{\EuScript M}\big(a\,;{\rm st}(\alpha)\big)}{=}{{\rm st}\big(t^{{}^*{\!\EuScript M}}(a\,;\alpha)\big)}
\end{fact}
\begin{proof}
  When a $t$ is function symbol of sort  ${\sf M}^{|x|}\to {\sf R}$, the claim holds by definition.
  When $t$ is a function symbol of sort ${\sf R}^{|y|}\to {\sf R}$, the claim follows from Fact~\ref{fact_terms_st}.
  Now, assume inductively that 

  \ceq{\hfill t_i^{\EuScript M}\big(a\,;{\rm st}(\alpha)\big)}{=}{{\rm st}\big(t_i^{{}^*{\!\EuScript M}}(a\,;\alpha)\big)}

  holds for the terms $t_1(x\,;y),\dots,t_n(x\,;y)$ and let $t=f(t_1,\dots,t_n)$ for some function $f$ of sort ${\sf R}^n\to {\sf R}$.
  Then the claim follows immediately from the induction hypothesis and Fact~\ref{fact_terms_st}.
\end{proof}

% Symbols in $L_{\sf R}$ have the interpretation they have in models.

\begin{lemma}\label{lem_st}
  Let ${}^*{\!\EuScript M}$ and ${\EuScript M}$ be as above.
  Then for every $\varphi(x\,;y)\in\mathds{I}$, \  $a\in M^{|x|}$ and $\alpha\in({}^*\!R)^{|y|}$ 
  
  \ceq{\hfill{}^*{\!\EuScript M}\models\varphi(a\,;\alpha)}
  {\Rightarrow}
  {{\EuScript M}\models\varphi\big(a\,;{\rm st}(\alpha)\big)}
\end{lemma}

\begin{proof}
  Suppose $\varphi(x\,;y)$ is $\mathds{I}$-atomic.
  If $\varphi(x\,;y)$ is a formula of $L_{\sf M}$ the claim is trivial. 
  Otherwise $\varphi(x\,;y)$ has the form $t(x\,;y)\in C$.
  Assume that the tuple $t(x\,;y)$ consists of a single term.
  The general case follows easily from this special case. 
  Assume that ${}^*{\!\EuScript M}\models t(a\,;\alpha)\in C$.
  Then ${\rm st}\big(t^{{}^*{\!\EuScript M}}(a\,;\alpha\big)\in C$ by Fact~\ref{fact_st1}.
  Therefore $t^{{\EuScript M}}(a\,;{\rm st}(\alpha)\in C$ follows from Fact~\ref{fact_st2}.
  This proves the lemma for $\mathds{I}$-atomic formulas.
  Induction is immediate. 
\end{proof}



%%%%%%%%%%%%%%%%%%%%%%%%%%%%%%%%%%%%
%%%%%%%%%%%%%%%%%%%%%%%%%%%%%%%%%%%%
%%%%%%%%%%%%%%%%%%%%%%%%%%%%%%%%%%%%
%%%%%%%%%%%%%%%%%%%%%%%%%%%%%%%%%%%%
%%%%%%%%%%%%%%%%%%%%%%%%%%%%%%%%%%%%
\section{Henson-Iovino approximations}\label{ultrapws}


For $\varphi,\varphi'\in\mathds{I}(M)$ (free variables are hidden) we write \emph{$\varphi'>\varphi$\/} if $\varphi'$ is obtained by replacing each atomic formula of the form $t\in C$ occurring in $\varphi$ with $t\in C'$ where $C'$ is some compact neighborhood of $C$.
If no such atomic formulas occur in $\varphi$, then $\varphi>\varphi$.
We also have $\varphi>\varphi$ when $\varphi=(t\in C)$ for some clopen set $C$.
Note that $>$ is a dense (pre)order of $\mathds{I}(M)$.
Formulas in (i) of Definition~\ref{def_LL} do not occur under the scope of a negation, therefore we always have that $\varphi\to\varphi'$.

We write \emph{$\tilde{\varphi}\perp\varphi$\/} when $\tilde{\varphi}$ is obtained by replacing each atomic formula $t\in C$ occurring in $\varphi$ with $t\in\tilde{C}$ where $\tilde{C}$ is some compact set disjoint from $C$.
Moreover the $\mathds{I}$-atomic formulas in $L_{\sf M}$ are replaced with their negation and each connective is replaced with its dual i.e., $\vee, \wedge, \exists, \forall, \existsR , \forallR $ are replaced with $\wedge,\vee,\forall,\exists,\forallR ,\existsR $ respectively.
We say that  $\tilde{\varphi}$ is a \emph{strong negation} of $\varphi$.
It is clear that $\tilde{\varphi}\rightarrow\neg\varphi$.

\begin{lemma}\label{lem_interpolation}
  For all  $\varphi\in\mathds{I}(M)$
  \begin{itemize}
    \item[1.]for every $\varphi'>\varphi$ there is a formula $\tilde{\varphi}\perp\varphi$ such that $\varphi\rightarrow\neg \tilde{\varphi}\rightarrow\varphi'$;
    \item[2.] for every\, $\tilde{\varphi}\perp\varphi$ there is a formula $\varphi'>\varphi$ such that  $\varphi\rightarrow\varphi'\rightarrow\neg \tilde{\varphi}$.
  \end{itemize}
\end{lemma}

\begin{proof}
  If $\varphi\in L$ the claims are obvious.
  Suppose $\varphi$ is of the form $t\in C$.
  Let $\varphi'$ be $t\in C'$, for some compact neighborhood of $C$.
  Let $O$ be an open set such that $C\subseteq O\subseteq C'$.
  Then $\tilde{\varphi}=(t\in R\smallsetminus O)$ is as required by the lemma.
  Suppose instead that $\tilde{\varphi}$ is of the form $t\in\tilde{C}$ for some compact $\tilde{C}$ disjoint from $C$.
  By the normality of $R$, there is  $C'$, a compact neighborhood of $C$ disjoint from $\tilde{C}$.
  Then  $\varphi'=\big(t\in C'\big)$ is as required.
  The lemma follows easily by induction.
\end{proof}


%%%%%%%%%%%%%%%%%%%%%%%%%%%%%%%%%%%%
%%%%%%%%%%%%%%%%%%%%%%%%%%%%%%%%%%%%
%%%%%%%%%%%%%%%%%%%%%%%%%%%%%%%%%%%%
%%%%%%%%%%%%%%%%%%%%%%%%%%%%%%%%%%%%
%%%%%%%%%%%%%%%%%%%%%%%%%%%%%%%%%%%%
\section{Morphisms}

\def\ceq#1#2#3{\parbox[t]{35ex}{$\displaystyle #1$}\parbox{5ex}{\hfil $#2$}{$\displaystyle #3$}}

Let ${\EuScript M}=\langle M,R\rangle$ and ${\EuScript N}=\langle N,R\rangle$ be two models.
We say that a partial map $f:M\to N$ is \emph{$\mathds{I}$-elementary\/} if for every $\varphi(x)\in\mathds{I}$ and every $a\in({\rm dom }f)^{|x|}$

\ceq{1.\hfill{\EuScript M}\models\varphi(a)}{\Rightarrow}{{\EuScript N}\models\varphi(fa).}

An $\mathds{I}$-elementary map that is total is called an \emph{$\mathds{I}$-elementary embedding.} 
% When the map ${\rm id}_M:M\hookrightarrow N$ is an $\mathds{I}$-embedding, that is, if for every  $\varphi(x)\in\mathds{I}$ and every $a\in M^{|x|}$
%
% \ceq{\hfill{\EuScript M}\models\varphi(a)}{\Rightarrow}{{\EuScript N}\models\varphi(a)},
%
We write \emph{${\EuScript M}\preceq^\mathds{I}{\EuScript N}$\/} and say that ${\EuScript M}$ is an \emph{$\mathds{I}$-elementary submodel\/} of ${\EuScript N}$. 

The definitions of \emph{$\mathds{H}$-elementary\/} map/embedding/submodel is obtained replacing $\mathds{I}$ by $\mathds{H}$.

As $\mathds{I}$ and $\mathds{H}$-elementary maps are in particular $L_{\sf M}$-elementary, they are injective.
However their inverse need not be $\mathds{I}$, respectively $\mathds{H}$-elementary.
In other words the converse implication in (1) may not hold.
We have choosen to work with the classical notion of truth at the cost of having this asymmetric notion of elementarity.
This contrast with~\cite{HI} where Henson and Iovino introduce a notion of approximated satisfaction.
This ensures that the corresponding morphisms (cf.\@ the approximated $\mathds{H}$-morphisms below) are invertible.

We say that the map $f:M\to N$ is \emph{approximately $\mathds{I}$-elementary\/} if for every formula $\varphi(x)\in\mathds{I}$, and every $a\in({\rm dom }f)^{|x|}$

\ceq{\hfill{\EuScript M}\models\varphi'(a)\ \textrm{ for all }\varphi'>\varphi}{\Rightarrow}{{\EuScript N}\models\varphi'(fa)\ \textrm{ for all }\varphi'>\varphi.}

We define \emph{approximately $\mathds{H}$-elementary\/} maps in the same manner.
We will see (Proposition~\ref{prop_approx}) that a slight amount of saturation sufficies to make these approximated versions of elementarity equivalent to their unapproximated version.
With full saturation, $\mathds{I}$ and $\mathds{H}$-elementarity coincide with ${\EuScript L}$-elementarity (Corollary~\ref{corol_Lcomplete}).

However, the following holds in general.

\begin{fact}
  If $f:M\to N$ is approximately $\mathds{I}$-elementary then

  \ceq{\hfill{\EuScript M}\models\varphi'(a)\textrm{ for all }\varphi'>\varphi}{\Leftrightarrow}{{\EuScript N}\models\varphi'(fa)\textrm{ for all }\varphi'>\varphi.}

  for every $\varphi(x)\in\mathds{I}$, and every $a\in({\rm dom }f)^{|x|}$.
  The same holds for approximately $\mathds{H}$-elementary maps.
\end{fact}

\begin{proof}
  Implication $\Leftarrow$ requires a proof.
  Assume the r.h.s.\@ of the equivalence.
  Fix $\varphi'>\varphi$ and prove ${\EuScript M}\models\varphi'(a)$.
  Let  $\varphi'>\varphi''>\varphi$.
  By Lemma~\ref{lem_interpolation} there is some $\tilde{\varphi}\perp\varphi''$ such that $\varphi''\rightarrow\neg\tilde{\varphi}\rightarrow\varphi'$.
  Then ${\EuScript N}\models\neg\tilde{\varphi}(fa)$ and therefore  ${\EuScript M}\models\neg\tilde{\varphi}(a)$.
  Then ${\EuScript M}\models\varphi'(a)$.
\end{proof}

\begin{comment}
Finally we introduce the morphisms that corrisponds to partial isomorphisms.
We say that $f:M\to N$ is a \emph{partial $\mathds{I}$-embedding\/} if (1) holds for all $\mathds{I}$-atomic formulas $\varphi(x)$.

\begin{fact}
  If $f:M\to N$ is a partial $\mathds{I}$-embedding then 

  \ceq{\hfill{\EuScript M}\models\varphi(a)}{\Leftrightarrow}{{\EuScript N}\models\varphi(fa)}

  for every $\mathds{I}$-atomic formula $\varphi(x)$ and every $a\in({\rm dom }f)^{|x|}$
\end{fact}

\begin{proof}
  The equivalence is trivial for formulas in $L_{\sf M}$ so we only consider $\mathds{I}$-atomic formulas as in (i) of Definion~\ref{def_0}.
  Implication $\Rightarrow$ holds by definition. 
  Vice versa, if ${\EuScript M}\models t(a)\notin C$ then, by normality,  ${\EuScript M}\models t(a)\in\tilde{C}$ for some compact $\tilde C$ disjoint of $C$.
  By the definition of $\mathds{I}$-embedding, ${\EuScript N}\models t(a)\in\tilde{C}$.
  Hence ${\EuScript N}\models t(a)\notin C$.
\end{proof}

\end{comment}


%%%%%%%%%%%%%%%%%%%%%%%%%%%%%%%%%%%%
%%%%%%%%%%%%%%%%%%%%%%%%%%%%%%%%%%%%
%%%%%%%%%%%%%%%%%%%%%%%%%%%%%%%%%%%%
%%%%%%%%%%%%%%%%%%%%%%%%%%%%%%%%%%%%
%%%%%%%%%%%%%%%%%%%%%%%%%%%%%%%%%%%%
\section{Compactness}

It is convenient to distinguish between consistency with respect to models and consistency with respect to ${\EuScript L}$-structures.
We say that a theory $T$ is \emph{${\EuScript L}$-consistent\/} when ${\EuScript M}\models T$ for some ${\EuScript L}$-structure ${\EuScript M}$.
We say that $T$ is \emph{consistent\/} when ${\EuScript M}$  is required to be a model.

\begin{theorem}\label{thm_compattezza}
  Let $T\subseteq\mathds{I}$ be finitely consistent.
  Then $T$ is consistent.
\end{theorem}

\begin{proof}
  Suppose $T$ is finitely consistent (or finitely ${\EuScript L}$-consistent, for that matter).
  By the classical compactness theorem ${}^*{\!\EuScript M}\models T$ for some ${\EuScript L}$-structure ${}^*{\!\EuScript M}=\langle M,{}^*\!R\rangle$.
  Let ${\EuScript M}=\langle M,R\rangle$ be the standard part of ${}^*{\!\EuScript M}$ as defined in Section~\ref{standard_part}.
  Then ${\EuScript N}\models T$ by Lemma~\ref{lem_st}.
\end{proof}

% \begin{corollary}\label{prop_tipi_compattezza}
%   Let $p(x\,;y)\subseteq\mathds{I}(M)$ be finitely consistent in ${\EuScript M}$, a model.
%   Then ${\EuScript N}\models\existsM x\,\exists^R y\,p(x\,;y)$ for some model ${\EuScript N}$ such that ${\EuScript M}\preceq^\mathds{I}{\EuScript N}$.
% \end{corollary}

% \noindent\llap{\dangersign\kern2ex}%


% \begin{definition}
%   We say that ${\EuScript M}$ is \emph{$\mathds{I}$-saturated\/} if for every $p(x)$ as in 1 and 2 below, ${\EuScript M}\models\existsM x\, p(a)$.
%   \begin{itemize}
%     \item[1.] $p(x)\subseteq\mathds{I}(A)$ for some $A\subseteq  M$ of cardinality $<|M|$ and $|x|=1$;
%     \item[2.] $p(x)$ is finitely satisfied in ${\EuScript M}$.
%   \end{itemize}
% \end{definition}
A model ${\EuScript N}$ is \emph{$\lambda$-$\mathds{I}$-saturated\/} if it realizes all types with fewer than $\lambda$ parameters that are finitely consistent in ${\EuScript N}$.
When $\lambda=|{\EuScript N}|$ we simply say  \emph{$\mathds{I}$-saturated.}
The existence of $\mathds{I}$-saturated models is obtained from the classical case just as for Theorem~\ref{thm_compattezza}.

\begin{theorem}
  Every model has an $\mathds{I}$-elementary extension to a saturated model (possibly of inaccessible cardinality).
\end{theorem}

% Homogeneity follows from saturation by the usual back-and-forth contruction.

% \begin{proposition}(Homogeneity)
%   Let ${\EuScript N}$ be saturated and of cardinality larger than $|\mathds{I}|$.
%   Then for every $ab\in M^\alpha$, where $\alpha<|N|$, such that $a\equiv b$, there is an $\mathds{I}$-automorphism of ${\EuScript N}$ that maps $a$ to $b$.
% \end{proposition}

The following proposition shows that a slight amount of saturation tames the $\mathds{I}$-formulas.


\begin{proposition}\label{prop_approx}
  Let ${\EuScript N}$ be an $\omega$-$\mathds{I}$-saturated model.
  Then the following holds in ${\EuScript N}$ for every formula $\varphi(x)\in\mathds{I}(N)$

  \ceq{\hfill\bigwedge_{\varphi'>\varphi}\varphi'(x\,;y)}
  {\leftrightarrow}
  {\varphi(x\,;y)}
\end{proposition}

\begin{proof}
  We prove $\rightarrow$, the non trivial implication.
  The claim is clear for atomic formulas.
  Induction for conjunction, disjunction and the universal quantifier is immediate.
 % 
  % Consider disjunction.
  % Let $\varphi=\varphi_i\vee\varphi_1$.
  % Assume inductively
 % 
  % \ceq{\hfill\varphi_i(x)}
  % {\leftrightarrow}
  % {\bigwedge_{\varphi_i'>\varphi_i}\varphi_i'(x)}
%
  % Then
%
  % \ceq{\hfill\varphi(x)}
  % {\leftrightarrow}
  % {\bigwedge_{\varphi_1'>\varphi_1}\varphi_1'(x) \ \vee \bigwedge_{\varphi_2'>\varphi_2}\varphi_2'(x)}
%
  % \ceq{}
  % {\leftrightarrow}
  % {\bigwedge_{\substack{\varphi_1'>\varphi_1\\ \varphi_2'>\varphi_2}}\varphi_1'(x) \vee\varphi_2'(x)}
%
  % \ceq{}
  % {\leftrightarrow}
  % {\bigwedge_{\varphi'>\varphi}\varphi'(x)}
%
  We consider case of the existential quantifiers of sort ${\sf M}$.
  The case of existential quantifiers of sort ${\sf R}$ is identical.
  Assume inductively
  
  \ceq{\textrm{ih.}\hfill\bigwedge_{\varphi'>\varphi}\varphi'(x,z\,;y)}
  {\rightarrow}
  {\varphi(x,z\,;y)}

  We need to prove

  \ceq{\hfill\bigwedge_{\varphi'>\varphi}\existsM z\,\varphi'(x,z\,;y)}
  {\rightarrow}
  {\existsM z\,\varphi(x,z\,;y)}

  From (ih) we have

  \ceq{\hfill\existsM z\,\bigwedge_{\varphi'>\varphi}\varphi'(x,z\,;y)}
  {\rightarrow}
  {\existsM z\,\varphi(x,z\,;y)}

  Therefore it suffices to prove

  \ceq{\hfill\bigwedge_{\varphi'>\varphi}\existsM z\,\varphi'(x,z\,;y)}
  {\rightarrow}
  {\existsM z\,\bigwedge_{\varphi'>\varphi}\varphi'(x,z\,;y)}

Replace $x,y$ with some fix but arbitrary parameters, say $a,\alpha$ and assume the antecedent, that is, the truth of the theory $\{\existsM z\,\varphi'(a,z\,;\alpha):\varphi'>\varphi\}$.
We need to prove the consistency of the type $\{\varphi'(a,z\,;\alpha):\varphi'>\varphi\}$.
By the saturation of ${\EuScript U}$, finite concistency suffices.
This is clear if we show that the antecedent is closed under conjunction.
Indeed it is easy to verify that if $\varphi_1,\varphi_2>\varphi$ then $\varphi_1\wedge\varphi_2>\varphi'$ for some $\varphi'>\varphi$.
In words, the set of approximations of $\varphi$ is a directed set.
%
% \noindent\llap{\textcolor{red}{\Large\danger}\kern1ex}\ignorespaces
% Consider the existential quantifiers of sort $R$
% Assume inductively
%
% \ceq{\hfill\bigwedge_{\varphi'>\varphi}\varphi'(x;y)}
% {\rightarrow}
% {\varphi(x;y)}
%
% As above, it suffices to prove
%
% \ceq{\#\hfill\bigwedge_{\varphi'>\varphi}\exists^C\! y\,\varphi'(x;y)}
% {\rightarrow}
% {\exists^C\! y\,\bigwedge_{\varphi'>\varphi}\varphi'(x;y)}
%
\end{proof}

\def\ceq#1#2#3{\parbox[t]{20ex}{$\displaystyle #1$}\parbox{5ex}{\hfil $#2$}{$\displaystyle #3$}}


For $p(x)\subseteq\mathds{I}(A)$, we write \emph{$p'(x)$\/} for the type

\ceq{\hfill p'(x)}{=}{\big\{\varphi'(x):\varphi'>\varphi\textrm{ for some }\varphi(x)\in p\big\}.}

The following consequence of the proposition above will be frequently used

\begin{remark}\label{remk_p'p}
  If evaluated in an $\omega$-$\mathds{I}$-saturated model, $p'(x)$ is equivalent to $p(x)$.
\end{remark}

%%%%%%%%%%%%%%%%%%%%%%%%%%%%%%%%%%%%
%%%%%%%%%%%%%%%%%%%%%%%%%%%%%%%%%%%%
%%%%%%%%%%%%%%%%%%%%%%%%%%%%%%%%%%%%
%%%%%%%%%%%%%%%%%%%%%%%%%%%%%%%%%%%%
%%%%%%%%%%%%%%%%%%%%%%%%%%%%%%%%%%%%
\section{The monster model}\label{monster}

We denote by \emph{${\EuScript U}=\langle U,R\rangle$\/} some large $\mathds{I}$-saturated structure which we call the \emph{monster model.}
For convenience we assume that the cardinality of ${\EuScript U}$ is an inaccessible cardinal which we denote by \emph{$\kappa$.}
Below we say \emph{model\/} for $\mathds{I}$-elementary submodel of ${\EuScript U}$.

Let $A\subseteq U$ be a small set throughout this section.
We define a topology on $U^{|x|}\times R^{|y|}$ which we call the \emph{$\mathds{I}(A)$-topology.}
The closed sets of this topology are the sets defined by the types $p(x\,;y)\subseteq\mathds{I}(A)$.
This is a compact topology by the $\mathds{I}$-compactness Theorem~\ref{thm_compattezza}.
The following fact demostrate how $\mathds{I}$-compactness applies in this context.
There are some subtle differences from the classical setting.
% \begin{lemma}
%   For every $\varphi(x\,;y)\in\mathds{I}(A)$ the set $\varphi(U\,;R)$ is a compact subset of $U^{|x|}\times R^{|y|}$.
% \end{lemma}

% \begin{proof}
%   We prove that every finitely intersecting family of closed subsets of $\varphi(U\,;R)$ has nonempty intersection.
%   Without loss of generality we can assume that the sets in this family have the form $\psi(U)\times C_1\times\dots\times C_{|y|}$ for some compact sets $C_i\subseteq R$ and some $\psi(x)\in\mathds{I}$.
%   Then by $\mathds{I}$-compactness the intersection of the family is nonempty.
% \end{proof}

\begin{fact}\label{fact_compactness_imp}
  Let $p(x)\subseteq\mathds{I}(A)$ be a type.
  Then for every  $\varphi(x)\in\mathds{I}(U)$
  \begin{itemize}
    \item[1.] if $p(x)\rightarrow\neg\varphi(x)$ then $\psi(x)\rightarrow\varphi(x)$ for some $\psi(x)$ conjunction of formulas in $p(x)$;
    \item[2.] if $p(x)\rightarrow\varphi(x)$ and $\varphi'>\varphi$ then $\psi(x)\rightarrow\varphi'(x)$ for some conjunction of formulas in $p(x)$.
  \end{itemize} 
\end{fact}

\begin{proof}
  Claim (1) is immediate by saturation.
  Claim (2) follows from the first by Lemma~\ref{lem_interpolation}.
\end{proof}



% \begin{example}
%   The equivalence in the proposition above does not extend to all formulas in $\mathds{I}$.
%   For a counter example, let both $M$ and $R$ are the real interval $[0,1]$.
%   The language has a function symbol ${\rm d}$ of sort $M^2\to R$.
%   Its interpretation is ${\rm d}^{\EuScript M}(x,y)=|x-y|$.
%   The formula 
  
%   \ceq{\hfill\varphi}{=}{\exists\varepsilon\big[\varepsilon\in\{0\}\ \wedge\ \existsM x,y\ \ {\rm d}(x,y)\cdot\varepsilon\in\{1\}\big]}

%   is false, though every $\varphi'>\varphi$ is true (in ${\EuScript M}$ and any $\mathds{I}$-extension of ${\EuScript M}$).\hfill\qedsymbol
% \end{example}


% The following fact is an immediate consequence.



% \begin{corollary}
%   If $\varphi(x),\psi(x)\in\mathds{I}(U)$ are mutually inconsistent then there are $\varphi'>\varphi$ and $\psi'>\psi$ that are mutually inconsistent.
% \end{corollary}





% \begin{fact}
%   Let $a\in M^{|x|}$ be such that ${\EuScript M}\not\models\varphi(a)$
%   Then, ${\EuScript M}\not\models\varphi'(a)$ for some $\varphi'>\varphi$.\hfill\qedsymbol
% \end{fact}














% \begin{fact}
%   Let $\varphi(x)\in\mathds{I}(A)$, where $A\subseteq M$.
%   Then for every $a\in M^{|x|}$ such that ${\EuScript M}\not\models\varphi(a)$ there is a formula $\tilde\varphi(x)\in\mathds{I}(A)$ such that ${\EuScript M}\models\tilde\varphi(a)$ and ${\EuScript M}\models\neg\existsM x\,[\tilde\varphi(x)\wedge\varphi(x)]$.
% \end{fact}


% \begin{fact}
%   For every $\varphi(x),\psi(x)\in\mathds{I}(U)$ such that $\neg\existsM x\,[\varphi(x)\wedge\psi(x)]$ there are $\varphi'>\varphi$ and $\psi'>\psi$ such that $\neg\existsM x\,[\varphi'(x)\wedge\psi'(x)]$.
% \end{fact}

% \begin{proof}
%   By Proposition~\ref{prop_approx} and saturation.
% \end{proof}

% The following are immediate consequences of the proposition above and Lemma~\ref{lem_interpolation}.

% \begin{corollary}\label{corol_Hcomplete1}
%   For every $a\in U^{|x|}$ and $\varphi(x)\in\mathds{H}(U)$, if $\neg\varphi(a)$ then $\tilde\varphi(a)$ for some $\tilde\varphi\perp\varphi$.\hfill\qedsymbol
% \end{corollary}


When $A\subseteq U$, we write \emph{$S_\mathds{I}(A)$\/} for the set of types 

\ceq{\hfill\mathds{I}\mbox{-tp}(a/A)}{=}{\big\{\varphi(x)\ :\ \varphi(x)\in\mathds{I}(A)\textrm{ such that }\varphi(a)\big\}}

as $a$ ranges over the tuples of elements of $U$.
We write  \emph{$S_{\mathds{I},x}(A)$\/} when the tuple of variables $x$ is fixed.
The same notation applies also with $\mathds{H}$ for $\mathds{I}$.

%  Similarly we define \emph{$S_\mathds{I}(A)$\/} and \emph{$S_{\mathds{I},x}(A)$.}
 The following proposition will be strengthen by Corollary~\ref{corol_Lcomplete} below.

 \begin{proposition}\label{prop_Hcomplete2}
   The types $p(x)\in S_\mathds{I}(A)$ are maximally consistent subsets of $\mathds{I}_x(A)$.
   That is, for every $\varphi(x)\in\mathds{I}(A)$, either $\varphi(x)\in p$ or $p(x)\rightarrow\neg\varphi(x)$.
   The same holds with $\mathds{H}$ for $\mathds{I}$.
 \end{proposition}

 \begin{proof}
  Let $p(x)=\mathds{I}\mbox{-tp}(a/A)$ and suppose $\varphi(x)\notin p$.
  Then $\neg\varphi(a)$.
  From Lemma~\ref{lem_interpolation} and Proposition~\ref{prop_approx} we obtain

  \ceq{\hfill\neg\varphi(x)}
  {\rightarrow}
  {\bigvee_{\tilde{\varphi}\perp\varphi}\tilde{\varphi}(x).}

  Hence $\tilde{\varphi}(a)$ holds for some $\tilde{\varphi}\perp\varphi$ and $p(x)\rightarrow\neg\varphi(x)$ follows.
 \end{proof}


% \begin{corollary}\label{corol_Hcomplete3}
%   The inverse of an $\mathds{H}$-elementary map $f:{\EuScript U}\to{\EuScript U}$ is $\mathds{H}$-elementary.
% \end{corollary}






%%%%%%%%%%%%%%%%%%%%%%%%%%%%%%%%%%%%
%%%%%%%%%%%%%%%%%%%%%%%%%%%%%%%%%%%%
%%%%%%%%%%%%%%%%%%%%%%%%%%%%%%%%%%%%
%%%%%%%%%%%%%%%%%%%%%%%%%%%%%%%%%%%%
%%%%%%%%%%%%%%%%%%%%%%%%%%%%%%%%%%%%
\section{Elimination of the quantifiers of sort ${\sf R}$}

We show that the quantifiers $\forallR $ and $\existsR $ can be eliminated up to some approximation.

\begin{proposition}
  The monster model (or any saturated model, for that matter) is $\mathds{H}$-homogeneous, that is, every $\mathds{H}$-elementary map $f:U\to U$ of cardinality $<\kappa$ extends to an automorphism.
\end{proposition}

\begin{proof}
  By Proposition~\ref{prop_Hcomplete2}, the inverse of an $\mathds{H}$-elementary map $f:U\to U$ is $\mathds{H}$-elementary.
  Then the usual proof by back-and-forth applies.
\end{proof}

As promised, we strengthen Proposition~\ref{prop_Hcomplete2}.
Let $A\subseteq U$ be a small set throughout this section.

\begin{corollary}\label{corol_Lcomplete}
  Let $p(x)\in S_\mathds{H}(A)$.
  Then $p(x)$ is complete for formulas in ${\EuScript L}_x(A)$.
  That is, for every $\varphi(x)\in{\EuScript L}(A)$, either $p(x)\rightarrow\varphi(x)$ or $p(x)\rightarrow\neg\varphi(x)$.
  Clearly, the same holds for $p'(x)$.
\end{corollary}

\begin{proof}
  If $b\models p(x)=\mathds{H}\mbox{-tp}(a/A)$ then there is an $\mathds{H}$-elementary map $f\supseteq{\rm id}_A$ such that $fa=b$.
  As $f$ exends to an automorphism and every automorphism is ${\EuScript L}$-elementary, the corollary follows.
\end{proof}

For $a,b\in U^{|x|}$ we write \emph{$a\equiv_Ab$\/} if $a$ and $b$ satisfy the same ${\EuScript L}$-formulas over $A$, bearing in mind that formulas in $\mathds{H}(A)$ or $\mathds{I}(A)$ suffices to test the equivalence.

In the classical setting, from equivalence of types one can derive equivalence of formulas.
Without negation, this is not true.
Still, we can infer an approximate form of equivalence.
Indeed, next proposition show that formulas in $\mathds{I}(A)$ are approximated by formulas in $\mathds{H}(A)$.

\begin{proposition}\label{prop_LHapprox1}
  Let $\varphi(x)\in\mathds{I}(A)$.
  For every given $\varphi'>\varphi$ there is some formula $\psi(x)\in\mathds{H}(A)$ such that $\varphi(x)\rightarrow\psi(x)\rightarrow\varphi'(x)$.
\end{proposition}

\begin{proof}
  By Corollary~\ref{corol_Lcomplete} and Remark~\ref{remk_p'p}

  \ceq{\hfill\neg\varphi(x)}{\rightarrow}{\bigvee_{p'(x)\rightarrow\neg\varphi(x)}p'(x)}

  where $p(x)$ ranges over $S_{\mathds{H},x}(A)$.
  By Fact~\ref{fact_compactness_imp} and Lemma~\ref{lem_interpolation}

  \ceq{\hfill\neg\varphi(x)}{\rightarrow}{\bigvee_{\neg\tilde{\psi}(x)\rightarrow\neg\varphi(x)}\neg\tilde{\psi}(x),}

  where $\tilde{\psi}(x)\in\mathds{H}(A)$.
  Equivalently,

  \ceq{\hfill\varphi(x)}{\leftarrow}{\bigwedge_{\tilde{\psi}(x)\leftarrow\varphi(x)}\tilde{\psi}(x).}

  By compactness, see Fact~\ref{fact_compactness_imp}, for every $\varphi'>\varphi$ there are some finitely many $\tilde{\psi}_i(x)\in\mathds{H}(A)$ such that

  \ceq{\hfill\varphi'(x)}{\leftarrow}{\bigwedge_{i=1,\dots,n}\tilde{\psi}_i(x)\ \ \leftarrow\ \ \varphi(x)}

  which yields the interpolant required by the proposition.
\end{proof}

We also need an approximation restult for the complement of formulas in $\mathds{I}(A)$.

\begin{proposition}\label{prop_LHapprox2}
  Let $\varphi(x)\in\mathds{I}(A)$ be such that $\neg\varphi(x)$ is consistent.
  Then  $\psi'(x)\rightarrow\neg\varphi(x)$ for some consistent $\psi(x)\in\mathds{H}(A)$ and some $\psi'>\psi$.
\end{proposition}


% \begin{proposition}\label{prop_LHapprox2}
%   Let $\varphi(x)\in\mathds{I}(A)$. For every given $\tilde{\varphi}\perp\varphi$ there is some $\psi(x)\in\mathds{H}(A)$ such that $\varphi(x)\rightarrow\psi(x)\rightarrow\neg\tilde{\varphi}(x)$.
% \end{proposition}

\begin{proof}
  Let $a\in U^{|x|}$ be such that $\neg\varphi(a)$.
  Let $p(x)=\mathds{H}\mbox{-tp}(a/A)$.
  By Corollary~\ref{corol_Lcomplete}, $p'(x)\rightarrow\neg\varphi(x)$. 
  By compactness  $\psi'(x)\rightarrow\neg\varphi(x)$ for some $\psi'>\psi\in p(x)$. 
\end{proof}


%%%%%%%%%%%%%%%%%%%%%%%%%%%%%%%%%%%%
%%%%%%%%%%%%%%%%%%%%%%%%%%%%%%%%%%%%
%%%%%%%%%%%%%%%%%%%%%%%%%%%%%%%%%%%%
%%%%%%%%%%%%%%%%%%%%%%%%%%%%%%%%%%%%
%%%%%%%%%%%%%%%%%%%%%%%%%%%%%%%%%%%%
\section{The Tarski-Vaught test}

The following proposition is not literally the Tarski-Vaught test.
Infact, it only applies when the larger structure is the moster model.

\begin{proposition}
  Let $M$ be a subset of $U$.
  Then the following are equivalent
  \begin{itemize}
    \item[1.] $M$ is the domain of a model ${\EuScript M}=\langle M,R\rangle$;
    \item[2.] for every formula $\varphi(x)\in\mathds{H}(M)$
    
    \noindent\kern-\leftmargin
    \ceq{\hfill\existsM x\,\varphi(x)}{\Rightarrow}
    {\textrm{ for every }\varphi'>\varphi\textrm{ there is an }a\in M\textrm{ such that }\varphi'(a);}
    \item[3.] for every formula $\varphi(x)\in\mathds{I}(M)$
    
    \noindent\kern-\leftmargin
    \ceq{\hfill\existsM x\,\neg\varphi(x)}{\Rightarrow}
    {\textrm{ there is an }a\in M\textrm{ such that }\neg\varphi(a).}
    
  \end{itemize}
\end{proposition}

\begin{proof}
  (1$\Rightarrow$2) Assume $\existsM x\,\varphi(x)$ and let $\varphi'>\varphi$ be given.
  By Lemma~\ref{lem_interpolation} there is some $\tilde{\varphi}\perp\varphi$ such that  $\varphi(x)\rightarrow\neg\tilde{\varphi}(x)\rightarrow\varphi'(x)$.
  Then $\neg\forallM x\,\tilde{\varphi}(x)$ hence, by (1), ${\EuScript M}\models\neg\forallM x\,\varphi(x)$.
  Then ${\EuScript M}\models\neg\tilde{\varphi}(a)$ for some $a\in M$. Hence ${\EuScript M}\models\varphi'(a)$ and $\varphi'(a)$ follows from (1).

  (2$\Rightarrow$3)
  Assume (2) and let $\varphi(x)\in\mathds{I}(M)$ be such that $\existsM x\,\neg\varphi(x)$.
  By Corollary~\ref{prop_LHapprox2}, there are a consistent $\psi(x)\in\mathds{H}(M)$ and some $\psi'>\psi$ such that $\psi'(x)\rightarrow\neg\varphi(x)$.
  Then (3) follows.

  (3$\Rightarrow$1)
  Assume (3).
  By the classical Tarski-Vaught test $M\preceq U$.
  Let ${\EuScript M}=\langle M,R\rangle$ be the unique  ${\EuScript L}$-structure that is a substructure of ${\EuScript U}$.
  Then  $\varphi(a)\ \Leftrightarrow\ {\EuScript M}\models\varphi(a)$ holds for every $a\in M^{|x|}$ and for every $\mathds{I}$-atomic formula $\varphi(x)$.
  Now, assume inductively
  
  \ceq{\hfill{\EuScript M}\models\varphi(a,b)}{\Rightarrow}{\varphi(a,b).}

  Using (3) and the induction hypothesis we prove by contrapposition that

  \ceq{\hfill{\EuScript M}\models\forallM y\,\varphi(a,y)}{\Rightarrow}{\forallM y\,\varphi(a,y)}.

  Indeed,

  \ceq{\hfill\neg\forallM y\,\varphi(a,y)}
  {\Rightarrow}{\existsM y\,\neg\varphi(a,y)}
  
  \ceq{}
  {\Rightarrow}
  {\neg\varphi(a,b)}\hfill for some $b\in M^{|y|}$\kern30ex

  \ceq{}
  {\Rightarrow}
  {{\EuScript M}\models\neg\varphi(a,b)}\hfill for some $b\in M^{|y|}$\kern30ex

  \ceq{}
  {\Rightarrow}
  {{\EuScript M}\not\models\forallM y\,\varphi(a,y).}

  Induction for the connectives $\vee$, $\wedge$, $\existsM $, $\existsR $, and $\forallR $ is straightforward.
\end{proof}

%%%%%%%%%%%%%%%%%%%%%%%%
%%%%%%%%%%%%%%%%%%%%%%%%
%%%%%%%%%%%%%%%%%%%%%%%%
%%%%%%%%%%%%%%%%%%%%%%%%
%%%%%%%%%%%%%%%%%%%%%%%%
\section{Entourages}

For $t(x,z)$ a term of sort ${\sf M}^{|x|+|z|}\to {\sf R}$ and $C\subseteq R^2$ we define 

\ceq{\hfill\tau_{t,C}(x,y)}{=}{\forallM z\ \langle t(x,z),t(y,z)\rangle\in C.}

As $C$ ranges over the compact neighborhoods of the diagonal $\Delta\subseteq R^2$ and $t(x,z)$ over the parameter-free terms, the formulas $\tau_{t,C}(x,y)$ define a system of entougages on $U^{|x|}$.
We refer to the topology associated as the \emph{${\sf R}$-topology.}
It is not difficult to verify that the ${\sf R}$-topology on $U^{|x|}$ coincides with the product of the ${\sf R}$-topology on $U$.
We define the type

\ceq{\hfill\emph{$x\sim_{\sf R}y$}}{=}{\Big\{\tau_{t,C}(x,y)\ :\ t(x,z)\textrm{ a term, } C\textrm{ a compact neigborhood of }\Delta\Big\}.}

In other words, the type above says that $\forallM z\ \big[t(x,z)=t(y,z)\big]$ holds for every term $t(x,z)$.
Again, it is easy to verify that if $|x|=|y|=\lambda$ then $x\sim_{\sf R}y$ if and only if $x_i\sim_{\sf R}y_i$ for every $i<\lambda$.



% We define the following equivalence relation on $U$

% \ceq{\hfill\emph{$a\sim_{\sf R}\!b$}}{=}{\forallM y\ \big[t(a,y)=t(b,y)\big]}\hfill 

% When two tuples of the same length $\bar a=\langle a_i:i<\lambda\rangle$ and $\bar b=\langle b_i:i<\lambda\rangle$ are such that $a_i\sim_{\sf R}\!b_i$ for every $i<\lambda$ we write \emph{$\bar a\sim_{\sf R}\bar b$.}


% \begin{fact} For $\bar a,\bar b\in U^{|x|}$
  
%   \ceq{\hfill\bar a\sim_{\sf R}\bar b}{\Leftrightarrow}{\forallM y\ \big[t(\bar a,y)=t(\bar b,y)\big]}\hfill for every term $t(x,y)$ of sort ${\sf M}^{|x|+|y|}\to {\sf R}$.
% \end{fact}

% \begin{proof}
%   By induction on $\lambda=|x|$ we prove that $t(\bar a,y)=t(\bar b,y)$ holds (universal quantification over the free variables is understood) for every term $t(x,y)$.
%   Assume inductively that
  
%   \ceq{\hfill t(\bar a,z,y)}{=}{t(\bar b,z,y)}

%   holds for every term $t(\bar x,z,y)$.   
%   In particular for any $a_\lambda$

%   \ceq{1.\hfill t(\bar a,a_\lambda,y)}{=}{t(\bar b,a_\lambda,y).}

%   If $a_\lambda\sim_{\sf R}\!b_\lambda$ then

%   \ceq{\hfill t(\bar x,a_\lambda,y)}{=}{t(\bar x,b_\lambda,y)}

%   and in particular 

%   \ceq{2.\hfill t(\bar b,a_\lambda,y)}{=}{t(\bar b,b_\lambda,y).}

%   From (1) and (2) we obtain

%   \ceq{\hfill t(\bar a,a_\lambda,y)}{=}{t(\bar b,b_\lambda,y).}

%   For limit ordinals induction is trivial.
% \end{proof}

Let $p(x)\subseteq\mathds{I}(A)$ for some small $A\subseteq U$.
We say that $p(x)$ is \emph{${\sf R}$-invariant\/} if $x\sim_{\sf R}\!y\rightarrow[p(x)\leftrightarrow p(y)]$.
It is immediate that $\mathds{I}$-formulas constructed without using (ii) of Definition~\ref{def_LL} are ${\sf R}$-invariant.

We say that $p(x)$ is a \emph{Cauchy type\/} if it is consistent and $p(x)\cup p(y)\rightarrow x\sim_{\sf R}\!y$.

\begin{fact}
  Let ${\EuScript M}$ be a model that is $\lambda$-$\mathds{I}$-saturated for some $\lambda>|{\EuScript L}|$.
  Then every ${\sf R}$-invariant Cauchy type $p(x)\subseteq\mathds{I}(M)$ is realized in ${\EuScript M}$.
\end{fact}

\begin{proof}
  Pick any $a\models p(x)$.
  For every formula $\varphi(x)$ in the type $a\sim_{\sf R}\!x$ and every $\varphi'>\varphi$ pick some $\psi(x)\in p$ such that $\psi(x)\rightarrow\varphi'(x)$.
  There are at most $|{\EuScript L}|$ such formulas $\psi(x)$.
  By saturation they have a common solution $b$ in $M$.
  Then $b\sim_{\sf R}a$ and, by invariance, $b\models p(x)$. 
\end{proof}

We say that a model ${\EuScript M}$ is \emph{Cauchy saturated\/} (or Cauchy complete) if every ${\sf R}$-invariant Cauchy type over $M$ is realized in ${\EuScript M}$.
Note that for any type $p(x)\subseteq\mathds{I}(M)$ the type $\bar p(x)=\exists y{\sim_{\sf R}}x\ p(y)$ is an ${\sf R}$-invariant.
If $p(x)$ is Cauchy also $\bar p(x)$ is Cauchy.
Therefore ${\EuScript M}$ is Cauchy saturated if every Cauchy type $p(x)\subseteq\mathds{I}(M)$ is realized by some $a\sim_{\sf R}M$.

\begin{comment}
The pr ${\EuScript M}$oof of the following proposition uses the same compactness argument used for Proposition~\ref{prop_LHapprox1}.

\begin{proposition}\label{prop_sim_approx}
  Let $\varphi(x)\in\mathds{I}(U)$ be ${\sf R}$-invariant.
  For every given $\varphi'>\varphi$ there are some $a_i\models\varphi(x)$, some terms $t_i(x,z)$, and some compact $C_i\subseteq R^2\smallsetminus\Delta$ such that 

  \ceq{\hfill\varphi(x)}{\rightarrow}{\bigwedge_{i=1,\dots,n}\tilde\tau_{t_i,C_i}(x,a_i)\ \ \rightarrow\ \ \varphi'(x),}

  where

  \ceq{\hfill\tilde\tau_{t,C}(x,z)}{=}{\existsM z\ \langle t(x,z),t(y,z)\rangle\in C.}

\end{proposition}

\begin{proof}
  As $\neg\varphi(x)$ is ${\sf R}$-invariant 

  \ceq{\hfill\neg\varphi(x)}{\leftrightarrow}{\bigvee_{a\,\models\neg\varphi(x)}\big\{\tau_{t,\Delta}(x,a)\ :\ t(x,z)\textrm{ a term}\big\}.}

  By Fact~\ref{fact_compactness_imp} and Lemma~\ref{lem_interpolation}

    \ceq{\hfill\neg\varphi(x)}{\leftrightarrow}{\bigvee_{\neg\tilde\tau_{t,C}(a,x)\rightarrow\neg\varphi(x)}\neg\tilde\tau_{t,C}(x,a),}

  where $C$ ranges over the compact subsetes of $R^2\smallsetminus\Delta$.
  Equivalently

  \ceq{\hfill\varphi(x)}{\leftrightarrow}{\bigwedge_{\varphi(x)\rightarrow\tilde\tau_{t,C}(x,a)}\tilde\tau_{t,C}(x,a).}

  By compactness, see Fact~\ref{fact_compactness_imp}, there are some some $a_i,t_i,C_i$ as in the lemma such that 

  \ceq{\hfill\varphi(x)}{\rightarrow}{\bigwedge_{i=1,\dots,n}\tilde\tau_{t_i,C_i}(x,a_i)\ \ \rightarrow\ \ \varphi'(x)}
\end{proof}
  
\end{comment}



%%%%%%%%%%%%%%%%%%%%%%%%%%%%%%%%%%%
%%%%%%%%%%%%%%%%%%%%%%%%%%%%%%%%%%%
%%%%%%%%%%%%%%%%%%%%%%%%%%%%%%%%%%%
%%%%%%%%%%%%%%%%%%%%%%%%%%%%%%%%%%%
%%%%%%%%%%%%%%%%%%%%%%%%%%%%%%%%%%%
%%%%%%%%%%%%%%%%%%%%%%%%%%%%%%%%%%%
\section{Example: metric spaces}

We discuss a simple example.
Let $M,d$ is a metric space.
Let $L_{\sf M}$ contain symbols only for continuous functions $M^n\to M$.
The language ${\EuScript L}$ has also a symbol $d$ for a function of sort ${\sf M}^2\to {\sf R}$.
Let $R=\mathds{R}\cup\{\pm\infty\}$ be the compactification of $\mathds{R}$.
Let ${\EuScript M}=\langle M,R\rangle$ be the structure that interprets the symbols of the language as natural.

Let ${\EuScript U}$ be a monster model, ${\EuScript M}\preceq^\mathds{I}{\EuScript U}$.
Clearly, $d$ does not define a metric on $U$ as there are pairs of elements at infinite distance.
However, when restricted to a ball of finite radius, $d$ defines a pseudometric on  $U$.
Therefore the notion of convergent sequence makes perfectly sense in $U$.

\begin{fact}
  Let $ {\EuScript U}$ be as above.
  Then for every model ${\EuScript N}=\langle N,R\rangle$ the following are equivalent for every $a\in U$
  \begin{itemize}
    \item[1.] $p(x)=\mathds{I}\mbox{-tp}(a/N)$ is a Cauchy type;
    \item[2.] there is a sequence $\langle a_i: i\in\omega\rangle$ of elements of $N$ that converges to $a$.
  \end{itemize} 
\end{fact}

\begin{proof}
  First note that, by the continuity of the terms with respect to the metric, $a\sim_{\sf R}\!b$ is equivalent to $d(a,b)=0$.

  (2$\Rightarrow$1) 
  Let $\langle \varepsilon_i: i\in\omega\rangle$ be a sequence of reals that converges to $0$ and such that $d(a_i,a)\le\varepsilon_i$ for every $i\in\omega$.
  Then the formulas $d(a_i,x)\le\varepsilon_i$ are in $p(x)$.
  Then every element realizing $p(x)$ is at distance $0$ from $a$.
  Therefore  $p(x)\rightarrow a\sim_{\sf R}\!x$.


  (1$\Rightarrow$2) 
  As $p(x)$ is Cauchy type, $p(x)\rightarrow a\sim_{\sf R}\!x$.
  Then $p'(x)\rightarrow d(a,x)<2^{-i}$ for all $i$.
  By compactness (see Fact~\ref{fact_compactness_imp}) there are formulas $\varphi_i(x)\in p$ such that $\varphi'_i(x)\rightarrow d(a,x)<2^{-i}$ for some $\varphi'_i>\varphi_i$.
  By $\mathds{I}$-elementarity there is an $a_i\in N$ such that $\varphi'(a_i)$.
  As $d(a,a_i)<2^{-i}$ for all $i$, the sequence $\langle a_i: i\in\omega\rangle$ converges to $a$.
\end{proof}


%%%%%%%%%%%%%%%%%%%%%%%%%%%%%%%%%%%
%%%%%%%%%%%%%%%%%%%%%%%%%%%%%%%%%%%
%%%%%%%%%%%%%%%%%%%%%%%%%%%%%%%%%%%
%%%%%%%%%%%%%%%%%%%%%%%%%%%%%%%%%%%
%%%%%%%%%%%%%%%%%%%%%%%%%%%%%%%%%%%
%%%%%%%%%%%%%%%%%%%%%%%%%%%%%%%%%%%
% \section{Continuous group actions}
% Let $M$ be a topological space.
% Let $G\le{\rm Homeo}(M)$ be a subgroup closed in the topology of the pointwise convergence.
% Then there is a compact topological space $R$ and a model ${\EuScript M}=\langle M,R\rangle$ such that ${\rm Aut}({\EuScript M})$ is isomorphic to $G$ as a topological group.

% Assume $M$ is compact, or otherwise substitute it with its Stone-Chech compactification.
% Consider $M^\omega$ as a topological space with the product topology.
% Let $G$ acts on $M^\omega$ coordinatewise. 
% Let $R$ be the set containing the topological closures of the $G$-orbits of elements of $M^\omega$, in notation this is the set $\big\{{\rm cl}(Ga):a\in M^\omega\big\}$.

% % First note that we can embed $M^{<\omega}$ into $M^\omega$, namely we map the tuple  $\langle a_1,\dots,a_n\rangle$, to the sequence  $\langle a_1,\dots,a_n^\omega\rangle$, where $a_n^\omega$ stands for an infinite sequence with constant value $a_n$.

% Let ${\EuScript M}=\langle M,R\rangle$.
% The language $L_{\sf M}$ is empty while ${\EuScript L}$ contains symbols for the functions

% \ceq{\hfill \pi^n\ :\ M^n}{\to}{R}

% \ceq{\hfill (a_1,\dots,a_n)}{\mapsto}{{\rm cl}\big(G(a_1,\dots,a_n^\omega)\big)},

% Where $a_n^\omega$ stands for the element $a_n$ repeated $\omega$-times.
% We claim that the ${\sf R}$-topology on $M$ coincides with its original topology.
% There is a natural embedding of $G$ into ${\rm Aut}({\EuScript M})$.
% Let $\gamma\in{\rm Aut}({\EuScript M})$ we claim that for every $a_1,\dots,a_n$ and every there is a $g\in G$ such that $\tau_{t,C}(\gamma(a),ga)$

%%%%%%%%%%%%%%%%%%%%%
%%%%%%%%%%%%%%%%%%%%%
%%%%%%%%%%%%%%%%%%%%%
%%%%%%%%%%%%%%%%%%%%%
%%%%%%%%%%%%%%%%%%%%%
%%%%%%%%%%%%%%%%%%%%%
%%%%%%%%%%%%%%%%%%%%%
\newcommand\biburl[1]{\url{#1}}
\BibSpec{arXiv}{%
  +{}{\PrintAuthors}{author}
  +{,}{ \textit}{title}
  +{}{ \parenthesize}{date}
  +{,}{ arXiv:}{eprint}
}
\begin{bibdiv}
\begin{biblist}[]\normalsize

\bib{HI}{article}{
   author={Henson, C. Ward},
   author={Iovino, Jos\'{e}},
   title={Ultraproducts in analysis},
   conference={
      title={Analysis and logic},
      address={Mons},
      date={1997},
   },
   book={
      series={London Math. Soc. Lecture Note Ser.},
      volume={262},
      publisher={Cambridge Univ. Press, Cambridge},
   },
   date={2002},
   pages={1--110},
  %  review={\MR{1967834}},
}

\bib{K}{arXiv}{
  author = {Keisler, H. Jerome},
  title = {Model Theory for Real-valued Structures},
  eprint={2005.11851},
  doi = {10.48550/ARXIV.2005.11851},
  url = {https://arxiv.org/abs/2005.11851},
  publisher = {arXiv},
  date = {2020},
}
\end{biblist}
\end{bibdiv}

%%%%%%%%%%%%%%%%%%%%%%%%%%%%%%%%%%%
%%%%%%%%%%%%%%%%%%%%%%%%%%%%%%%%%%%
%%%%%%%%%%%%%%%%%%%%%%%%%%%%%%%%%%%
%%%%%%%%%%%%%%%%%%%%%%%%%%%%%%%%%%%
%%%%%%%%%%%%%%%%%%%%%%%%%%%%%%%%%%%
%%%%%%%%%%%%%%%%%%%%%%%%%%%%%%%%%%%
\section{Measure spaces}

\noindent\llap{\textcolor{red}{\Large\dangersign}\kern1ex}\ignorespaces
Garbage after this point

Let ${\EuScript B}$ be a $\sigma$-algebra of subsets of $\Omega$.
Let $F$ be the set of $\mathds{R}$-valued ${\EuScript B}$-simple functions, i.e.\@ linear combinations of indicator functions of sets in ${\EuScript B}$.
Let $M$ be the set of $\mathds{R}$-valued ${\EuScript B}$-measures concentrated on finitely many points of $\Omega$.

Let ${\EuScript M}=\langle M,F,R\rangle$, where $R=\mathds{R}\cup\{\pm\infty\}$ is the compactification of $\mathds{R}$.
The sort ${\sf M}$ in the previous section is replaced here by two sorts which we call ${\sf M}$ and ${\sf F}$.
Here $L_{\sf MF}$ is the language of the structure $\langle M,F\rangle$.
As both $F$ and $M$ are lattice $\mathds{R}$-algebras, we assume that $L_{\sf MF}$ has symbols for these lattice algebras.
There is also a function symbol of sort ${\sf F}\times{\sf M}\to{\sf M}$ that is interpreted as the natural action of functions on measures (via multiplication).
Finally, in ${\EuScript L}$ there is a function symbol ${\rm I}$ of sort ${\sf M}\to{\sf R}$ that is interpreted as the integral over $\Omega$.

Let ${\EuScript U}=\langle {}^*\!M,{}^*\!F,R\rangle$ be a monster model such that ${\EuScript M}\preceq^\mathds{I}{\EuScript U}$.

\def\ceq#1#2#3{\parbox[t]{25ex}{$\displaystyle #1$}\parbox{5ex}{\hfil $#2$}{$\displaystyle #3$}}

Let $\mu$ be a $\mathds{R}$-valued ${\EuScript B}$-measure.
By saturation, there is some ${}^*\!\mu\in{}^*\!M$ such that $g\,{\cdot}{}^*\!\mu=\int_\Omega g\,{\rm d}\mu$ for every $g\in F$.

Let $\gamma:\Omega\to\mathds{R}$. 
By saturation, there is some ${}^*\!\gamma\in{}^*\!F$ such that ${}^*\!\gamma\mathbin\cdot m=\gamma\mathbin\cdot m$ for every $m\in M$. 
Note that, as $m$ has finite support, $\gamma\mathbin\cdot m$ is a well-defined element of $M$ for all $\gamma$.

For $X\in{\EuScript B}$ we write ${\rm I}_X(\,\mbox{-}\,)$ for ${\rm I}(1_X\cdot\,\mbox{-}\,)$.

\begin{lemma}
Assume $\gamma$ is measurable.
Then for every $X\in{\EuScript B}$

\ceq{\hfill {\rm I}_X\big({}^*\!\gamma\cdot{}^*\!\mu\big)}{=}{\int_X \gamma\,{\rm d}\mu}


\end{lemma}

\begin{proof}
  Assume for simplicity that both $\gamma$ and $\mu$ are non-negative.
  % The general case follows by decomposition.
  Let $G\subseteq R$ be the following set

  \ceq{\hfill G}{=} {\bigg\{\int_X g\,{\rm d}\mu\ :\ 0\le g\le\gamma,\ g\in F\bigg\}.}

  If $G$ is bounded, pick for every $\varepsilon\in\mathds{R}^+$ some $\alpha_{\varepsilon}\in G$ such that 

  \ceq{\hfill\int_X \gamma\,{\rm d}\mu}{\le}{\alpha_{\varepsilon,f}+\varepsilon.}
 
  If $G_f$ is unbounded, let $\alpha_{\varepsilon,f}=+\infty$ for all $\varepsilon$.
  Let $y$ be a variable of sort ${\sf M}$.
  Let $p(y)$ be the $\mathds{I}$-type that contains, for all $f$ as in the lemma, the formulas 

  \ceq{\hfill{\rm I}\big( f{\cdot} y\big)}{\ge}{\alpha}
  \hfill for all $\alpha\in G_f$;

  \ceq{\hfill{\rm I}\big( f{\cdot} y\big)}{\le}{\alpha_{\varepsilon,f}+\varepsilon}
  \hfill for all $\varepsilon\in\mathds{R}^+$\cdot.

  By basic measure theory, $p(y)$ is finitely consistent in ${\EuScript M}$.
  Any realization of $p(y)$ in ${\EuScript U}$ proves the lemma.
\end{proof}


In $M$ and $F$ we use the functions $\,\mbox{-}^+$, $\,\mbox{-}^-$, $\,|\mbox{-}|$, and the relation $\le$ as usually defined in lattice vector spaces.
We use the symbol $0$ to denote the zeros of $M$, $F$, and $R$.
We denote by $1\in F$ the function that is identically $1\in R$.

The $\mathds{I}$-formula $\forall^{\sf F}\!f\ge0\,\ {\rm I}\big(f{\cdot}\mu^+\big)\ge0$ is obviously true in ${\EuScript M}$ and so is the same formula for $\mu^-$.
As the action of $F$ on $M$ distributes with the algebraic operations of $M$ and $F$, properties such as the Hahn decomposition are immediate in ${\EuScript M}$ and are inherited by $\EuScript U$.

Note that the $L_{\sf M}$-formula $\forall^{\sf F}\!f\!\in[0,1]\,\ f^2{=}\,f$ defines the indicator functions. 

For $\nu,\mu\in{}^*\!M$ we write $\nu\ll\mu$ for the ${\EuScript L}$-formula 

\hfil$\forallR \varepsilon\ \existsR \delta\ \forall^{\sf F}\!f\ \big[{\rm I}\big(|f{\cdot}\mu|\big)<\delta\rightarrow {\rm I}\big(|f{\cdot}\nu|\big)<\varepsilon\big]$.

\begin{proposition} 
  Let $\nu,\mu\in{}^*\!M$ be bounded and non-negative such that  $\nu\ll\mu$.
  Then there is a $g\in{}^*\!F$ such that $|\nu-g\mu|\sim_R0$.
\end{proposition}

\begin{proof}
  The following $\mathds{I}$-formula is true in ${\EuScript M}$.
  
  $
\forall\mu,\nu\ \forall\varepsilon>0\ \Big[\forall^{\sf F}\!f\ {\rm I}\big(|f{\cdot}\nu|\dotminus|f{\cdot}\mu|\big)<\varepsilon \rightarrow \exists^{\sf F}\!g\ \forall^{\sf F}\! f\ {\rm I}\big(f{\cdot}|\nu-g{\cdot}\mu|\big)\le\varepsilon\Big]$


\end{proof}

\begin{theorem}[Radon-Nikodym] 
  Let $\nu,\mu$ be bounded non-negative ${\EuScript B}$-measures on $\Omega$ such that $\nu\ll\mu$.
  Then there is a ${\EuScript B}$-measurable $f:\Omega\to\mathds{R}$ such that 
  $$
  \int_X f{\rm d}\mu = \int_X {\rm d}\nu
  $$
  for every $X\in{\EuScript B}$.
\end{theorem}
\end{document}
%%%%%%%%%%%%%%%%%%%%%%%%
%%%%%%%%%%%%%%%%%%%%%%%%
%%%%%%%%%%%%%%%%%%%%%%%%
%%%%%%%%%%%%%%%%%%%%%%%%
%%%%%%%%%%%%%%%%%%%%%%%%
\section{Equivalences}


We define an equivalence relation \emph{$(\sim_{\EuScript U})$\/} on $ U$ as follows

\ceq{1.\hfill a\,\sim_{\EuScript U}b}
{\Leftrightarrow}
{\Big({\EuScript U}\models t(a)= t(b)}
\ \ for every $t(x)\in\mathds{T}(U)\Big)$,

where $|x|=|a|=|b|=1$.

Hence $a\,\sim_{\EuScript U}b$ if and only if ${\rm d}_t(a,b)=0$ for all the pseudometrics ${\rm d}_t(\mbox{-},\mbox{-})$ defined below. 

\begin{definition}\label{def_uniformity}
  Let $x$ be a single variable.
  Let $t(x)\in\mathds{T}(U)^{<\omega}$ be a finte tuple of terms, say $t(x)=t_1(x),\dots,t_n(x)$.
  We define a pseudometric ${\rm d}_t(\mbox{-},\mbox{-})$ on $U$ as follows

  \ceq{\hfill\emph{${\rm d}_t(a,b)$}}{=}{\max_{i=1,\dots,n}\big\{|t_i^{\EuScript U}(a)-t_i^{\EuScript U}(b)|\big\}}

\end{definition}

We say that $a\in U$ is close to $M$ if 


\ceq{\hfill a\,\equiv^\mathds{I}_{\EuScript M}x}
{\rightarrow}
{a\sim_{\EuScript U}x},


\begin{proposition}
  The following are equivalent for every $a\in U$
  \begin{itemize}
    \item[1.] $a$ is close to ${\EuScript M}$;
    \item[2.] there is a net of elements of $M$ that converges to $a$. 
  \end{itemize}
\end{proposition}

\begin{proof}
  Let $p(x)={\rm tp}_\mathds{I}(a/U)$.

  (1$\Rightarrow$2) \ 
  The set of metrics ${\rm d}_t$ in Definion~\ref{def_uniformity} is a directed set under the relation of refinement.
  For every such metric ${\rm d}_t$ we have that

  \ceq{\hfill p_{\restriction M}(x)}{\rightarrow}{{\rm d}_t(x,a)\le0}

  By Lemma~\ref{lem_compactness_implication}, for every $\varepsilon>0$ there some consistent $\varphi(x)\in p$ such that

  \ceq{\hfill \varphi(x)}{\rightarrow}{{\rm d}_t(x,a)<\varepsilon}

  As $\varphi(x)$ is consistent in ${\EuScript M}$, there is some $b_{t,\varepsilon}\in M$ such that ${\rm d}_t(b_{t,\varepsilon},a)<\varepsilon$.
  This defines a net of elements of $M$ that converges to $a$.

  (2$\Rightarrow$1) \ 
  Let $I$ be a directed set.
  Let $\langle b_i:i\in I\rangle$ be a net of elements of $M$ that converges to $a$.
  Then, $c\sim_{\EuScript U} a$ for every $c\models p_{\restriction M}(x)$.
  Hence $p_{\restriction M}(x)\rightarrow p(x)$.
\end{proof}

\end{document}


\clearpage

  The language $L_{\sf M}$ and its interpretation are subject to the following conditions.
  \begin{itemize}
  \item[1.] Functions only have one of the following sorts (for any $m,n\in\omega$)
  \begin{itemize}
    \item[a.] $\mathds{R}^n\to\mathds{R}$
    \item[b.] $ M^n\times{M}^m\to\textrm{any of }  M,\ M,\textrm{ or }\mathds{R}$
    % \item[c.] $ M^n\times{M}^m\to M$
    % \item[d.] $ M^n\times{M}^m\to\mathds{R}$
  \end{itemize}
  \item[2.] There is a symbol for every (total) uniformly continuous function $\mathds{R}^n\to\mathds{R}$.
  \item[3.] The interpretation of functions symbolsof sort $M^n\to\mathds{R}$ have \textit{bounded\/} range.
  \item[4.] Every element of $M$ is the image of some term of sort $ M^n\to M$.  
  \item[5.] There is only one unary predicate: $x\le0$, of sort $\mathds{R}$.
  \end{itemize}
  Below, when the sort $M$ is not displaied in the definitions, we assume that $M=M$ and that the language contains the identity map ${\rm id}: M\to M$.
  In other words, that $M$ is redundant.


We call $M$ the \emph{unit ball\/} of ${\EuScript M}$: there is where the action takes place.
The auxiliary sort ${M}$ can be ignored at a first reading.
Is is introduced as it makes the description of many important examples less contrived, e.g.\@ Example~\ref{ex_banach} below.

\begin{definition}
  We write \emph{$\mathds{T}(A)$\/} for the set of terms of sort $ M^n\times\mathds{R}^m\to\mathds{R}$ with parameters in some $A\subseteq  M$.
  Up to equivalence, these terms have the form $f(t(x),y)$, where $t(x)$ is a tuple of terms of sort $M^{|x|}\to\mathds{R}$, and $f$ is a function $\mathds{R}^{|t|+|y|}\to\mathds{R}$.
  Note that the terms $\mathds{T}(A)$ have bounded range.
\end{definition}

Now define a subset of the set first order formulas with parameters in a set $A\subseteq M$.
Note that we do not allow negation, nor quantification over $M$.
Also, equality among elements of $M$ is excluded.

\begin{definition}
  We define the set of formulas \emph{$\mathds{I}(A)$\/} inductively
  \begin{itemize}
  \item[i.] $\mathds{I}(A)$ contains the atomic formula $t\le0$ for every term $t\in\mathds{T}(A)$.
  \item[ii.] It is closed under the Boolean connectives $\bot$, $\top$, $\wedge$, $\vee$, and the quantifiers $\forallM$, $\existsM $ of sort $ M$.
  \item[iii.] It is closed under the quantifier $\forallM$ of sort $\mathds{R}$ relativized to any definable subset of $\mathds{R}$.
  \end{itemize}
\end{definition}

Note that in iii of the definition above, definability is intended with respect to $L(M)$, the full first order language of ${\EuScript M}$, which is strictly larger than $\mathds{I}(A)$.
In particular, if $\varphi(x,\varepsilon)\in\mathds{I}(A)$, also $\forall\varepsilon{>}0\;\varphi(x,\varepsilon)\in\mathds{I}(A)$.

\begin{example}[Banach spaces]\label{ex_banach}
  Given a Banach space  $V$ we define a structure ${\EuScript M}=\langle  M,M,\mathds{R}\rangle$ as follows.
  Let $M=V$ and let $M=\{a\in M: \|a\|\le1\}$ be the closed unit ball of $V$.
  Besides the symbols mentioned above, $\mathds{I}$ contains a function symbol for the natural embedding ${\rm id}: M\to M$.
  It also contains a symbol for the norm $\|\mbox-\|:M\to\mathds{R}$.
  Finally, $\mathds{I}$ contains the usual symbols of the language of vector spaces.
  These have sort ${M}^n\to M$, for the appropiate $n\in\{0,1,2\}$.

  Terms of sort $M^{|x|}\to\mathds{R}$ have the form $f(t(x))$ where $f$ is a uniformly continuous function and

  \ceq{\hfill t(x)}{=}{\Big\|\sum_{i=1}^{|x|}\lambda_i x_i\,\Big\|.}

  Therefore, as for all $a\in M^{|x|}$

  \ceq{\hfill t^{{\EuScript M}}(a)}{\le}{\sum_{i=1}^{|x|}|\lambda_i|,}

  condition 3 of Definition~\ref{def_LL} is  satisfied.
  Conditions 4 is immediate.
\end{example}


\begin{example}[Metric spaces of finite diameter]\label{ex_metric}
  Given a metric space $(M,{\rm d})$ of finite diameter we define a structure ${\EuScript M}=\langle  M,\mathds{R}\rangle$ as follows.
  The only function symbols is ${\rm d}$.
  Note that 3 is satisfied because ${\rm d}$ is bounded.
\end{example}

%%%%%%%%%%%%%%%%%%%
%%%%%%%%%%%%%%%%%%%
%%%%%%%%%%%%%%%%%%%
%%%%%%%%%%%%%%%%%%%
\section{$\mathds{I}$-relations}\label{L-relations}

Equality between elements in the unit ball is not among the formulas that Definition~\ref{def_LL} enumerates in $\mathds{I}$.
Logic without equality is not uncommon in mathematics.
For instance, in the theory of integration equality beween functions is not significative notion.
Of course, there is a quick remedy to this.
One can work with equivalence classes of functions that coincide almost everywhere.
In this way, equality regains it meaning.
But this has some drawbacks.
In fact, this makes impossible to speak of the value that a function takes at a point and there are situations when this is inconvenient.

In our general setting, for similar reasons, structures that contain irrelevat copies of equivalent objects may offer a better grasp than terse quotiented structures.

Without equality, relations, rather than partial maps, are the natural tool to discuss elementarity. 

Let ${\EuScript M}$ and ${\EuScript N}$ be models.
We say that $R\subseteq M\times N$ is an \emph{$\mathds{I}$-(elementary) relation\/} between ${\EuScript M}$ and ${\EuScript N}$ (or on ${\EuScript M}$ if the two coicide) if for every $\varphi(x)\in\mathds{I}$

\ceq{\hfill{\EuScript M}\models\varphi(a)}{\Leftrightarrow}{{\EuScript N}\models\varphi(b)} \hfill for every $a$ and $b$ such that $a\mathbin{R}b$.

Recall that, when $a=a_1,\dots,a_n$ and $b=b_1,\dots,b_n$ are tuples, $a\mathbin{R}b$ stands for $a_i\mathbin{R}b_i$ for every $i\in\{1,\dots,n\}$.

We define an equivalence relation \emph{$(\sim_{\EuScript M})$\/} on $ M$ as follows

\ceq{1.\hfill a\,\sim_{\EuScript M\!}b}
{\Leftrightarrow}
{\Big({\EuScript M}\models\varphi(a)\leftrightarrow\varphi(b)}
\ \ for every $\varphi(x)\in\mathds{I}( M)\Big)$,

where $|x|=|a|=|b|=1$.
Note that this relation would be trivial had we included  in $\mathds{I}(M)$ equality between elements of $M$.

The following proposition is easily proved by induction on the syntax. 

\begin{proposition}
  The following are equivalent for every $a,b\in M$.
  \begin{itemize}
    \item[1.] $a\equiv_{\EuScript M}b$;
    \item[2.] ${\EuScript M}\models t(a)=t(b)$ for every $t(x)\in\mathds{T}(M)$, with $|x|=1$.
  \end{itemize}
\end{proposition}

\begin{lemma}
  The relation $(\sim_{\EuScript M\!})\subseteq  M^2$ is an $\mathds{I}$-relation.
  Moreover, it is maximal among the $\mathds{I}$-relations on ${\EuScript M}$, i.e.\@ no $\mathds{I}$-relation properly contains $(\sim_{\EuScript M\!})$.
\end{lemma}

\begin{proof}
  Assume $a\sim_{\EuScript M\!}b$, where $a=a_1,\dots,a_n$ and $b=b_1,\dots,b_n$.
  Recall that this means that $a_i\sim_{\EuScript M\!}b_i$ for all $i\in\{1,\dots,n\}$.
  Let $\Delta$ denote the diagonal relation on $ M$.
  Note that $a_i\sim_{\EuScript M\!}b_i$ is equivalent to saying that $\Delta\cup\{(a_i,b_i)\}$ is an $\mathds{I}$-relation.
  As $\mathds{I}$-relations are closed under composition $\Delta\cup\big\{(a_1,b_1),\dots,(a_n,b_n)\big\}$ is $\mathds{I}$-elementary.
  It follows that for every $\varphi(x)\in\mathds{I}$

  \ceq{2.\hfill{\EuScript M}\models\varphi(a)}{\Leftrightarrow}{{\EuScript M}\models\varphi(b).}
  
  This proves that $(\sim_{\EuScript M\!})$ is an $\mathds{I}$-relation.
  Finally, maximality is immediate.
  % Suppose $E$ and equivalence $\mathds{I}$-relation on ${\EuScript M\!}$ properly containing $(\sim_{\EuScript M\!})$.
  % Pick $a,b\in M$ such that $a\mathbin{E}b$ and $a\not\sim_{\EuScript M\!}b$.
\end{proof}


\begin{lemma}
  Let $R\subseteq  M\times N$ be total and surjective $\mathds{I}$-relation.
  Then there is a unique maximal $\mathds{I}$-relation containing $R$.
  This maximal $\mathds{I}$-relation is equal to both $(\sim_{\EuScript M\!})\,R$ and $R\,(\sim_{\EuScript N\!})$, where justapposition of relations stands for composition.
\end{lemma}
\begin{proof}
  It is immediate to verify that $(\sim_{\EuScript M\!})\,R$ is an $\mathds{I}$-relation containing $R$.
  Let $S$ be any maximal $\mathds{I}$-relation containing $R$.
  By maximality, $(\sim_{\EuScript M\!})\,S=S$.
  As $S$ is a total relation $(\sim_{\EuScript M\!})\subseteq SS^{-1}$.
  Therefore, by the lemma above, $(\sim_{\EuScript M\!})=SS^{-1}$.
  As $R$ is a surjective relation, $S\subseteq S\,S^{-1}R$.
  Finally, by maximality, we conclude that $S=(\sim_{\EuScript M\!})\,R$.
  A similar argument proves that $S=R\,(\sim_{\EuScript N\!})$.
\end{proof}

We write ${\rm Aut}({\EuScript M})$ for the set of maximal, total and surjective, $\mathds{I}$-relations $R\subseteq  M^2$.
The choice of the symbol Aut is motivated by the lemma above.
In fact any such relation $R$ induces a unique automorphism on the (properly defined) quotient structure ${\EuScript M}/{\sim_{\EuScript M\!}}$.

% Thought in most situations one could dispense with ${\EuScript M}$ in favour of ${\EuScript M}/{\sim_{\EuScript M\!}}$, in concrete cases one has a better grip on the first than on the latter.
% Therefore below we insist in working with $\mathds{I}$-relations in place of $\mathds{I}$-elementary maps.


%%%%%%%%%%%%%%%%%%%%%%%%%%%%%%%
%%%%%%%%%%%%%%%%%%%%%%%%%%%%%%%
%%%%%%%%%%%%%%%%%%%%%%%%%%%%%
%%%%%%%%%%%%%%%%%%%%%%%%%%%%%
%%%%%%%%%%%%%%%%%%%%%%%%%%%%%
\section{Elementary substructures}

An \emph{$\mathds{I}$-(elementary) embedding\/} is an $\mathds{I}$-relation that is functional, injective and total.

When $\varnothing$ is an $\mathds{I}$-relation between ${\EuScript M}$ and ${\EuScript N}$ we say that ${\EuScript M}$ and ${\EuScript N}$ are \emph{$\mathds{I}$-(elementary) equivalent\/} and write \emph{${\EuScript M}\equiv^\mathds{I} {\EuScript N}$.}
When $A\subseteq M\cap N$, we write \emph{${\EuScript M}\equiv_{\!A}{\EuScript N}$\/} if $\Delta_{\!A}$, the diagonal relation on $A$, is an $\mathds{I}$-relation.
In words, we say that ${\EuScript M}$ and ${\EuScript N}$ are $\mathds{I}$-equivalent \emph{over $A$.}
Finally, we write \emph{${\EuScript M}\preceq^\mathds{I}{\EuScript N}$\/} when ${\EuScript M}\subseteq{\EuScript N}$, i.e.\@ ${\EuScript M}$ is a substructure of ${\EuScript N}$, and ${\EuScript M}\equiv_M^\mathds{I}{\EuScript N}$.
In words, we say that ${\EuScript M}$ is an \emph{$\mathds{I}$-(elementary) substructure\/} of ${\EuScript N}$.


Let $A\subseteq  M$.
The \emph{$\mathds{I}(A)$-topology\/} on $M^{|x|}$ is the topology that has, as a base of closed sets, the $\mathds{I}(A)$-definable sets.

The $\mathds{I}(A)$-topology is not T$_0$.
The following lemma asserts that to obtain a Hausdorff topology it suffices to quotient by $\equiv_A^\mathds{I}$.
This is the so-called the Kolmogorov quotient.

We need the following definitions that will be relevant in many of the proofs below.
If $\varphi(x)\in L(A)$ and $\varepsilon>0$ we write \emph{${}^{\varepsilon\kern-2pt}\varphi(x)$\/} for the formula obtained by replacing every occurence in $\varphi$ of the atomic formula $t(x)\le0$ with $t(x)-\varepsilon\le0$ or, less pedantically, $t(x)\le\varepsilon$.
We write \emph{${}^{\neg\varepsilon\kern-2pt}\varphi(x)$\/} for the formula obtained by replacing $t(x)\le0$ with $\varepsilon\le t(x)$.

Note that, if $\varphi(x)\in\mathds{I}(A)$ then ${}^{\varepsilon\kern-2pt}\varphi(x),\ {}^{\neg\varepsilon\kern-2pt}\varphi(x)\in\mathds{I}(A)$.
Moreover, the positivity of the formulas in $\mathds{I}(A)$ ensures that

\ceq{\hfill\varphi(x)}{\to}{{}^{\varepsilon\kern-2pt}\varphi(x)}

\ceq{\textrm{and}\hfill {}^{\neg\varepsilon\kern-2pt}\varphi(x)}{\to}{\neg\varphi(x)}

The following fact is immediate.

\begin{fact}\label{fact_immediate}
  For every $a\in M^{|x|}$ and very $\varphi(x)\in\mathds{I}(A)$
  
  \ceq{\hfill{\EuScript M}\ \models\ \forall\varepsilon > 0\ {}^{\phantom{\neg}\varepsilon\kern-2pt}\varphi(a)}{\Leftrightarrow}{{\EuScript M}\ \models\ \varphi(a)}
  
  \ceq{\hfill{\EuScript M}\ \models\ \exists\varepsilon > 0\ {}^{\neg\varepsilon\kern-2pt}\varphi(a)}{\Leftrightarrow}{{\EuScript M}\ \models\ \neg\varphi(a)}\hfill\qedsymbol
\end{fact}

\begin{lemma}
  If $a\nequiv_A^\mathds{I}b$ are two elements of $M^{|x|}$ then there are two formulas $\varphi(x), \psi(x)\in\mathds{I}(A)$ such that $\varphi(a)$ and $\psi(b)$ hold and $\varphi(M)\cap\psi(M)=\varnothing$.
\end{lemma}

\begin{proof}
  Let $\varphi(x)\in\mathds{I}(A)$ be such that $\varphi(a)$ and $\neg\varphi(b)$.
  By the fact above ${}^{\neg\varepsilon\kern-2pt}\varphi(b)$ holds for some $\varepsilon>0$.
  Then let $\psi(x)={}^{\neg\varepsilon\kern-2pt}\varphi(x)$.
\end{proof}

\begin{proposition}[Tarski-Vaught test]\label{prop_Tarski-Vaught} Let ${\EuScript M}\subseteq{\EuScript N}$.
  Let $x$ be a single variable.
  Then the following are equivalent
  \begin{itemize}
    \item[1.] ${\EuScript M}\preceq^\mathds{I}{\EuScript N}$;
    % \item[2.] $M$ is dense in $N$ w.r.t.\@ the $\mathds{I}(M)$-topology on $N$;
    \item[2.] for every $\varphi(x)\in \mathds{I}(M)$, if $\varphi(N)\neq\varnothing$, then ${\EuScript N}\models\varphi(b)$ for some $b\in M$.
  \end{itemize}
\end{proposition}
\begin{proof}
  % Equivalence 2$\Leftrightarrow$3 is tautological.

  

  Implication 1$\Rightarrow$2 is clear.
  As for 2$\Rightarrow$1, we prove by induction on the syntax of $\varphi(z)$, where $z$ is a finite tuple, that

  \ceq{3.\hfill {\EuScript N}\models\varphi(a)}{\Leftrightarrow}{{\EuScript M}\models\varphi(a)} \hfill for every $a\in  M^{|z|}$

  For $\varphi(z)$ atomic 3 follows from ${\EuScript M}\subseteq{\EuScript N}$.
  Induction for the connectives $\wedge$ and $\vee$ is straightforward.
  Induction for the existential quantifier is as in the classical case.
  We prove the case of the universal quantifier over the unit ball.
  Assume that 3 holds for the formula $\psi(z,y)$ and prove that it holds for the formula $\forallM y\,\psi(z,y)$.
  The the implication $\Rightarrow$ is immediate.
  To prove $\Leftarrow$, note that

  \ceq{\hfill{\EuScript N}\models\neg\forallM y\ \psi(a,y)}{\Rightarrow}{{\EuScript N}\models\forallM y\,{}^{\neg\varepsilon}\psi(a,y)} \hfill for some $\varepsilon>0$

  \ceq{}{\Rightarrow}{{\EuScript M}\models\forallM y\,{}^{\neg\varepsilon\kern-.2pt}\psi(a,y)}\hfill for some $\varepsilon>0$

  \ceq{}{\Rightarrow}{{\EuScript M}\models\neg\forallM y\ \psi(a,y)}.

  Induction for the universal quantifier of sort $\mathds{R}$ is immediate.
\end{proof}

% \begin{theorem}[Downward L\"owenheim-Skolem] For every $A\subseteq N$ there is a model $A\subseteq  M\preceq N$ of cardinality $\le |L(A)|$.
% \end{theorem}
% \begin{proof} (Needs some checking.)
%   For every set $A$ there is a base for the $A\mbox{-}$topology that has cardinality $\le |L(A)|$. A set $ M$ of the required cardinality that is dense in the  $ M\mbox{-}$topology is obtained as in the classical downward L\"owenheim-Skolem theorem.
% \end{proof}



%%%%%%%%%%%%%%%%%%%%%%%%%%%%%%%%%%%%
%%%%%%%%%%%%%%%%%%%%%%%%%%%%%%%%%%%%
%%%%%%%%%%%%%%%%%%%%%%%%%%%%%%%%%%%%
%%%%%%%%%%%%%%%%%%%%%%%%%%%%%%%%%%%%
%%%%%%%%%%%%%%%%%%%%%%%%%%%%%%%%%%%%
\section{Ultraproducts}\label{ultrapws}

\def\ceq#1#2#3{\parbox[t]{20ex}{$\displaystyle #1$}\parbox{5ex}{\hfil $#2$}{$\displaystyle #3$}}


We recall some standard definitions about limits.
Let $I$ be a non-empty set.
Let $F$ be a filter on $I$.
If $f:I\to{\mathds R}$ and $\lambda\in{\mathds R}\cup\{\pm\infty\}$ we write

\hfil$\displaystyle \lim_{i\to F}f(i)=\lambda$

if $f^{-1}[A]\in F$ for every $A\subseteq{\mathds R}\cup\{\pm\infty\}$ that is a neighborhood of $\lambda$.
Such a $\lambda$ is unique and, when $F$ is an ultrafilter, it always exists.
When $f$ is bounded, $\lambda\in\mathds{R}$.

Let $I$ be an infinite set.
Let $\langle {\EuScript M}_i:i\in I\rangle$ be a sequence of structures, say  ${\EuScript M}_i=\langle  M_i,M_i,\mathds{R}\rangle$, that are \emph{uniformly bounded,} that is, the bounds in 3 of Definition~\ref{def_LL} are the same for all ${\EuScript M}_i$.

Let $F$ be an ultrafilter on $I$.

\begin{definition}\label{def_ultraproduct}
  We define a structure \emph{${\EuScript N}=\langle  N,N,\mathds{R}\rangle$\/} that we call the \emph{ultraproduct\/} of the models $\langle{\EuScript M}_i:i\in I\rangle$.
  \begin{itemize}
    \item[1.] $N$ comprise the sequences $\hat a: I\to\bigcup_{i\in I} M_i$ such that $\hat a\,i\in  M_i$.
    \item[2.] $N$ comprise the sequences $t^{\EuScript N}(\hat a)$ of the form $t^{{\EuScript M}_i}(\hat ai)$, where $t(x)$ is a term of sort $ M^n\to M$.
    \item[3.] If $f$ is a function of sort $ M^n\times{M}^m\to  M$ then $f^{\EuScript N}(\hat a,t^{{\EuScript N}}\big(\hat c)\big)$ is the sequence 
    $f^{{\EuScript M}_i}\big(\hat ai,t^{{\EuScript M}_i}(\hat ci)\big)$.
    \item[4.] Similarly when $f$ is of sort $ M^n\times{M}^m\to M$.
    \item[5.] If $f$ is a function of sort $ M^n\times{M}^m\to\mathds{R}$ then 
    $$
    f^{\EuScript N}\big(\hat a,t^{\EuScript N}(\hat c)\big)\ =\ \lim_{i\to F}f^{{\EuScript M}_i}\big(\hat ai,t^{{\EuScript M}_i}(\hat ci)\big).
    $$ 
  \end{itemize}
  As usual, if ${\EuScript M}_i={\EuScript M}$ for all $i\in I$, we say that ${\EuScript N}$ is an \emph{ultrapower\/} of ${\EuScript M}$.
\end{definition}

The limit in 5 of the definition above always exists because $F$ is an ultrafilter.
It is finite because all models ${\EuScript M}_i$ have the same bounds.

The following fact is easily proved by induction on the syntax.

\begin{fact}
  For every term $t(x)$ of sort $ M^n\to\mathds{R}$  
  
  \ceq{\hfill t^{\EuScript N}\big(\hat a\big)}{=}{\lim_{i\to F}t^{{\EuScript M}_i}\big(\hat ai\big).}\hfill\qedsymbol
\end{fact}

Finally, we prove 

\begin{proposition}[\L\v{o}\'s Theorem]\label{thm_Los}
  Let ${\EuScript N}$ be as above and let $\varphi(x,y)\in\mathds{I}(A)$.
  Let $\hat a\in N^{|x|}$.
  Then there is a function $u_{-}:\mathds{R}^+\to F$ such that for every $\lambda\in\mathds{R}^{|y|}$

  \ceq{1.\hfill {\EuScript N}\models\varphi(\hat a, \lambda)}
  {\Rightarrow}
  {u_\varepsilon\subseteq\big\{ i\in I\ :\ {\EuScript M}_i\models{}^{\varepsilon\kern-2pt}\varphi(\hat ai,\lambda)\big\} \ \textrm{for every }\varepsilon>0.}

  \ceq{2.\hfill {\EuScript N}\not\models\varphi(\hat a, \lambda)}
  {\Rightarrow}
  {u_\varepsilon\subseteq\big\{ i\in I\ :\ {\EuScript M}_i\not\models{}^{\varepsilon\kern-2pt}\varphi(\hat ai,\lambda)\big\} \ \textrm{for some }\varepsilon>0.}

\end{proposition}
\begin{proof}
  Suppose that $\varphi(x,y)$ is atomic, say

  \ceq{\hfill\varphi(x,y)}
  {=}
  {f(t(x),\,y)\le0,} 
  
  where $t$ is a tuple of terms of sort  $ M^{|x|}\to\mathds{R}$ and $f$ is a uniformly continuous function $\mathds{R}^{|t|+|y|}\to\mathds{R}$. 


  Let $\beta'\in\mathds{R}^{|t|}$ be such that
  $\displaystyle\lim_{i\to F}\ t(\hat ai)=\beta'$.

  Given $\varepsilon>0$, let $B_\varepsilon$ be a neighborhood of $\beta'$ such that 
  
  \ceq{}{\ }{\big|f(\beta,\,\lambda)-f(\beta',\lambda)\big|<\varepsilon}\hfill for every $\beta\in B_\varepsilon$ and every $\lambda\in\mathds{R}^{|y|}$.
  
  Such a neighborhood exists by the uniform continuity of $f$.
  Finally, to obtain 1 and 2 above it suffices to define 

  \ceq{\hfill u_\varepsilon}{=}{\big\{i\ :\  t(\hat a i)\in B_\varepsilon\big\}.}

  The rest of the inductive proof is straightforward and is left to the reader.
\end{proof}


%%%%%%%%%%%%%%%%%%%%%%%%%%%%%%%%%%%%
%%%%%%%%%%%%%%%%%%%%%%%%%%%%%%%%%%%%
%%%%%%%%%%%%%%%%%%%%%%%%%%%%%%%%%%%%
%%%%%%%%%%%%%%%%%%%%%%%%%%%%%%%%%%%%
%%%%%%%%%%%%%%%%%%%%%%%%%%%%%%%%%%%%
\section{Saturation}


Let $p(x)\subseteq\mathds{I}(M)$, where $x$ has the sort of the unit ball.
We say that $p(x)$ is \emph{finitely satisfied in ${\EuScript M}$\/} if for every conjunction of formulas in $p(x)$, say $\varphi(x)$, and for every $\varepsilon>0$, there is an $a\in  M^{|x|}$ such that ${\EuScript M}\models{}^{\varepsilon\kern-2pt}\varphi(a)$.

The following definition is completely standard

\begin{definition}
  We say that ${\EuScript M}$ is \emph{$\mathds{I}$-saturated\/} if for every $p(x)$ as in 1 and 2 below, there is an $a\in  M^{|x|}$ such that ${\EuScript M}\models p(a)$
  \begin{itemize}
    \item[1.] $p(x)\subseteq\mathds{I}(A)$ for some $A\subseteq  M$ of cardinality $<|M|$ and $|x|=1$;
    \item[2.] $p(x)$ is finitely satisfied in ${\EuScript M}$.
  \end{itemize}
\end{definition}

Now we prove that every model embeds $\mathds{I}$-elementarily in an $\mathds{I}$-saturated one.
First we prove the following lemma.

\begin{lemma}\label{lem_compattezza}
  Every model ${\EuScript M}$ embeds $\mathds{I}$-elementarily in a model ${\EuScript N}$ that realizes all types as in the definition above.
\end{lemma}

\begin{proof}
  Consider the collection of types such that 1 and 2 above.
  Assume that each type has a distinct variable and let $x$ be the concatenation of all these variables.
  We denote by $p(x)$ the union of all these types.
  Let $I$ be the set of formulas $\xi(x)$ that are satisfied in ${\EuScript M}$.
  For every formula $\varphi(x)$ and every $\varepsilon>0$ define $X_{\varphi,\varepsilon}\subseteq I$ as follows

  \ceq{\hfill X_{\varphi,\varepsilon}}{=}{\Big\{\xi(x)\in I\ :\ \xi(M)\ \subseteq\ {}^{\varepsilon\kern-2pt}\varphi(M)\Big\}}

  Note that ${}^{\varepsilon\kern-2pt}\varphi(x)$ is satisfied in ${\EuScript M}$ if and only if $X_\varphi\neq\varnothing$ if and only if ${}^{\varepsilon\kern-2pt}\varphi(x)\in X_{\varphi,\varepsilon}$.
  Moreover 
  
  \ceq{\hfill X_{\varphi\wedge\psi,\ \min\{\varepsilon,\delta\}}}{\subseteq}{X_{\varphi,\varepsilon}\cap X_{\psi,\delta}.}
  
  Then, as $p(x)$ is finitely consistent, the set 
  
  \ceq{\hfill B}{=}{\big\{X_{\varphi,\varepsilon}\,:\,\varphi(x)\in p\textrm{ and }\varepsilon>0\big\}}
  
  has the finite intersection property.
  Extend $B$ to an ultrafilter $F$ on $I$.
  Let ${\EuScript N}$ be the ultrapower of ${\EuScript M}$ over $F$.
  That is, the model with unit ball $N=M^I$ obtained as in Definition~\ref{def_ultraproduct}.
  By \L\v o\'s Theorem, ${\EuScript M}$ embeds elementariy into ${\EuScript N}$.
  It remains to prove that ${\EuScript N}$ realizes $p(x)$.

  For every formula $\xi(x)\in I$ choose some $a_\xi\in  M^{|x|}$ such that $\xi(a_\xi)$.
  %For legibility we conflate $(M^I)^{|x|}$ with $(M^{|x|})^I$.
  Let $\hat a\in (M^{|x|})^I$ be the function that maps $\xi(x)\mapsto a_\xi$.
  We claim that ${\EuScript N}\models\varphi(\hat a)$ for every $\varphi(x)\in p$.
  By Fact~\ref{fact_immediate}, it suffices to prove that ${\EuScript N}\models{}^{\varepsilon\kern-2pt}\varphi(\hat a)$ for every $\varepsilon>0$.
  Suppose not, then by \L\v o\'s Theorem (Proposition~\ref{thm_Los} with $\varepsilon$ for $\lambda$), for some $\varepsilon'>0$ the set 

  \ceq{\hfill Y_{\varphi,\varepsilon+\varepsilon'}}{=}{\Big\{\xi(x)\in I\ :\ {\EuScript M}\not\models{}^{\varepsilon+\varepsilon'\kern-2pt}\varphi(a_\xi)\Big\}}

  belongs to $F$.
  This is a contradiction because $B\subseteq F$ contains the set $X_{\varphi,\varepsilon+\varepsilon'}\subseteq Y_{\varphi,\varepsilon+\varepsilon'}$. 
\end{proof}

\begin{corollary}\label{thm_compattezza}
  Every model ${\EuScript M}$ is an $\mathds{I}$-elementary subtructure of some saturated model ${\EuScript N}$ (possibly of inaccessible cardinality).
\end{corollary}

We denote by \emph{${\EuScript U}$\/} some large $\mathds{I}$-saturated structure which we call the \emph{monster model.}
The unit ball of ${\EuScript U}$ is denoted by \emph{$U$.}
The cardinality of ${\EuScript U}$ is an inaccessible cardinal that we denote by \emph{$\kappa$.}
Below we say \emph{model\/} for $\mathds{I}$-elementariy substructure of ${\EuScript U}$.


\begin{lemma}\label{lem_compactness_implication}
  Let $p(x)\subseteq\mathds{I}(A)$, where $A\subseteq U$ has small cardinality.
  Let $\varphi(x)\in\mathds{I}(U)$ be such that $p(x)\rightarrow\varphi(x)$.
  Then for every $\varepsilon>0$ there is a formula $\psi(x)$, conjunction of formulas in $p(x)$, such that $\psi(x)\rightarrow\neg{}^{\neg\varepsilon\kern-2pt}\varphi(x)$.
\end{lemma}

\begin{proof}
  Let $\varepsilon>0$ be given.
  Suppose that $\psi(x)\wedge{}^{\neg\varepsilon\kern-2pt}\varphi(x)$ is consistent for every $\psi(x)$ that is conjunction of formulas in $p(x)$.
  By saturation there is a realization of $p(x)\cup{}^{\neg\varepsilon\kern-2pt}\varphi(x)$.
  By Fact~\ref{fact_immediate}, $p(x)\wedge\neg\varphi(x)$ is also consistent.
\end{proof}

%%%%%%%%%%%%%%%%%%%%%%%
%%%%%%%%%%%%%%%%%%%%%%%
%%%%%%%%%%%%%%%%%%%%%%%
%%%%%%%%%%%%%%%%%%%%%%%
%%%%%%%%%%%%%%%%%%%%%%%
%%%%%%%%%%%%%%%%%%%%%%%
\section{Complete models}

\def\ceq#1#2#3{\parbox[t]{25ex}{$\displaystyle #1$}\parbox{5ex}{\hfil $#2$}{$\displaystyle #3$}}

Let ${\EuScript M}$ be a model.
Let $a\in U^{|x|}$.
We say that ${\EuScript M}$ \emph{defines $a$ in the limit\/} if 

\ceq{\hfill p_{\restriction M}(x)}{\rightarrow}{p(x):={\rm tp}_\mathds{I}(a/U)}

or, in other words, if $a'\equiv^\mathds{I}_Ma$ implies $a'\sim_{\EuScript U}a$.

Note that the definability of a tuple in the limit is equivalent to the definability of its components.
This follows from the following obvious identity

\ceq{\hfill {\rm tp}_\mathds{I}(a,b/U)}
{=}
{{\rm tp}_\mathds{I}(a/U)\ \cup\ {\rm tp}_\mathds{I}(b/U).}

Now we characterize definability in the limit in topological terms.


\begin{definition}
  A model ${\EuScript M}$ is complete if it contains, up to $(\sim_{\EuScript U})$-equivalence, all points it defines in the limit.
\end{definition}

Let \emph{$\bar M$\/} be the set points that are defined in the limit by ${\EuScript M}$.
We denote by \emph{$\bar{\EuScript M}$\/} the (unique) substructure of ${\EuScript U}$ with unit ball $\bar M$.
We call  $\bar{\EuScript M}$ the \emph{completion\/} of ${\EuScript M}$.

We say that a model ${\EuScript N}$ is $M$-$\mathds{I}$-saturated if it realizes all constistent types $p(x)\subseteq\mathds{I}(M)$.
The following proposition is easy to prove.

\begin{proposition}
  $\bar{\EuScript M}$ is the intersection of all $M$-$\mathds{I}$-saturated models.
\end{proposition}

\begin{proposition}
  $\bar{\EuScript M}$ is complete.
\end{proposition}

\begin{proof}
  Suppose $a\in U$ is defined in the limit by $\bar{\EuScript M}$, that is

  \ceq{\hfill p'(x,b)}{\rightarrow}{p(x):={\rm tp}_\mathds{I}(a/U)}
  
  for some $b\in\bar M^{|z|}$ and $p'(x,z)\subseteq\mathds{I}(M)$ such that $p_{\restriction\bar M}(x)=p'(x,b)$.  
  Let $q(z)={\rm tp}_\mathds{I}(b/U)$, then

  \ceq{\hfill q_{\restriction M}(z)\ \cup\ p'(x,z)}{\rightarrow}{p(x)}
  
  Finally, $a\in\bar M$ follows because

  \ceq{\hfill p_{\restriction M}(x)}{\rightarrow}{\existsM z\ \big[q_{\restriction M}(z)\ \cup\ p'(x,z)\big].}
\end{proof}

\begin{question}
  $\bar{\EuScript M}\preceq_\mathds{I}{\EuScript U}$ \ ?
\end{question}
% %%%%%%%%%%%%%%%%%%%%%%%
% %%%%%%%%%%%%%%%%%%%%%%%
% %%%%%%%%%%%%%%%%%%%%%%%
% %%%%%%%%%%%%%%%%%%%%%%%


% Let ${\EuScript M}$ be a substructure of ${\EuScript U}$.
% Let $a\in U^{|x|}$.
% We say that ${\EuScript M}$ \emph{defines $a$ in the limit\/} if there is a formula $\varphi(x)\in\mathds{I}(M)$ and some infinite sets $B_1,\dots,B_n\subseteq M$ such that

% \ceq{\hfill\big\{\varphi(x)\big\}\ \cup\ p_{\restriction C}(x)}{\rightarrow}{p(x):={\rm tp}_\mathds{I}(a/U)}

% for every $C$ that has infinite intersection with all $B_1,\dots,B_n$.



% The following proposition is an immediate consequence of the definition of the relation $(\sim_{\EuScript U})$ in Section~\ref{L-relations}.

% \begin{proposition}
%   Suppose that $a'\equiv^\mathds{I}_Ma$ are both defined in the limit by ${\EuScript M}$.
%   Then $a'\sim_{\EuScript U}a$.
% \end{proposition}

% Finally we define completeness as a weak form of saturation.
% It is easy to verify that if ${\EuScript M}$ is the metric space in Example~\ref{ex_metric}, then the completeness of ${\EuScript M}$ is equivalent to the Cauchy-completeness of $(M,{\rm d})$.

% The proof of the following proposition is tedious but straightforward.

% \begin{proposition}
%   $\bar{\EuScript M}$ is complete.
% \end{proposition}

% \begin{proof}
%   Let $a\in U$ be defined in the limit by $\bar{\EuScript M}$.
%   For some formula $\varphi(x)\in\mathds{I}(\bar M)$ and some infinite set $B\subseteq\bar M$ we have that for every infinite $C\subseteq B$

%   \ceq{\hfill\big\{\varphi(x)\big\}\ \cup\ p_{\restriction C}(x)}{\rightarrow}{p(x):={\rm tp}_\mathds{I}(a/U)}

%   Let $b$ be a tuple that enumerates $B$ and the parameters of  $\varphi(x)$.
%   We rewrite the condition above with the dependence from $b$ displaied

%   \ceq{\hfill\big\{\varphi(x,b)\big\}\ \cup\ p_{\restriction C}(x,b)}{\rightarrow}{p(x,b).}

%   where $p(x,z)={\rm tp}_\mathds{I}(a,b/U)$.
%   As $b$ is defined in the limit by ${\EuScript M}$, there is a 

%   \ceq{\hfill\big\{\psi(z)\big\}\ \cup\ q_{\restriction C}(z)}{\rightarrow}{q(z)={\rm tp}_\mathds{I}(b/U)}
% \end{proof}

% Without additional hypothesis we cannot guarantee that $\bar{\EuScript M}$ is a model.

% \begin{theorem}
%   $\bar{\EuScript M}$ is an existentially closed substructure of ${\EuScript U}$.
% \end{theorem}

% \begin{proposition}
%   $\bar{\EuScript M}\preceq^\mathds{I}{\EuScript U}$.
% \end{proposition}

% \begin{proof}
%   Apply the Tarski-Vaught Test~\ref{prop_Tarski-Vaught}.
%   Let $\psi(x,z)\in\mathds{I}$ and $a\in\bar M^{|z|}$ be such that $\psi(x,a)$ is consistent.
% Let $p(z)={\rm tp}_\mathds{I}(a/U)$.
% For some $B\subseteq M$ and some formula $\varphi(z)\in\mathds{I}(M)$ 

%   \ceq{\hfill\big\{\varphi(z)\big\}\ \cup\ p_{\restriction C}(z)}{\rightarrow}{\existsM x\ \psi(x, z)}
% \end{proof}



%%%%%%%%%%%%%%%%%%%%%%%%%%%
%%%%%%%%%%%%%%%%%%%%%%%%%%%
%%%%%%%%%%%%%%%%%%%%%%%%%%%
%%%%%%%%%%%%%%%%%%%%%%%%%%%
%%%%%%%%%%%%%%%%%%%%%%%%%%%
\section{Homogeneity}
Recall that $\kappa$,  the cardinality of ${\EuScript U}$, is an inaccessible cardinal.

\begin{fact}
  Let $R\subseteq{\EuScript U}^2$ be an $\mathds{I}$-relation of cardinality $<\kappa$.
  Then there is a total and surjective $\mathds{I}$-relation $S\subseteq{\EuScript U}^2$ containing $R$.
\end{fact}

\begin{proof}
  We apply the usual back-and-forth construction with a pinch of extra caution.
  Let $a$ be an enumeration of the domain of $R$.
  Let $\bar a=\langle a_i:i<\lambda\rangle$ be an enumeration of all tuples of length $|a|$ such that $aRa_i$.
  As $\kappa$ is inaccessible, $\lambda<\kappa$.
  Let $b\in U$.
  It suffices to prove that there is a $c$ such that $R\cup\{\langle b,c\rangle\}$ is an $\mathds{I}$-relation.
  Let $p(x,z)={\rm tp}(b,a)$ and let
  
  \ceq{\hfill q(x,\bar z)}{=}{\bigcup_{i<\lambda}p(x,z_i).}

  We claim that $q(x,\bar a)$ is a finitely consistent type.
  A finite conjunction of formulas in $q(x,\bar a)$ has the form $\psi(x,a_{i_1})\wedge\dots\wedge\psi(x,a_{i_n})$.
  As $\psi(b,a)$ and $a_{i_1},\dots,a_{i_n}\,R\;a,\dots,a$, we conclude that the condition $\psi(x,a_{i_1})\wedge\dots\wedge\psi(x,a_{i_n})$ is satisfied.
  The existence of the required element $c$ follows by saturation.
\end{proof}

\begin{corollary}
  Let $A\subseteq{\EuScript U}$ have cardinality $<\kappa$.
  Let $p(x)={\rm tp}_\mathds{I}(a/A)$, where $a\in{\EuScript U}^{|x|}$ is a tuple of length $|x|<\kappa$.
  Then

  \ceq{\hfill p({\EuScript U})}{=}{\big\{b : bRa,\ R\in{\rm Aut}({\EuScript U}/A)\big\}}
\end{corollary}


% The following corollary of \L\v{o}\'s Theorem is identical to its classical counterpart.

% \begin{corollary}
%   For every model ${\EuScript M}$ and every ultrafilter $F$ on $I$, an infinite set, let ${\EuScript N}$ be corresponding ultrapower of ${\EuScript M}$.
%   Then there is an $\mathds{I}$-elementary embedding of ${\EuScript M}$ in ${\EuScript N}$.\qed
% \end{corollary}

% Note also that, unlike the classical ultraproduct, here we do not quotient the structure obtained in Definition~\ref{def_ultraproduct}.
% However, we note the following fact which is an easy consequence of \L\v o\'s Theorem.

% \begin{fact}
%   Let $R\subseteq N^2$ be the set of those pairs $(\hat a, \hat b)$ such that $\{i\in I:\hat ai=\hat bi\}\in F$.
%   Then $R$ is an $\mathds{I}$-relation.\qed
% \end{fact}

\end{document}


%%%%%%%%%%%%%%%%%%%%
%%%%%%%%%%%%%%%%%%%%
%%%%%%%%%%%%%%%%%%%%
%%%%%%%%%%%%%%%%%%%%
%%%%%%%%%%%%%%%%%%%%
\section{Examples}

\def\ceq#1#2#3{\parbox[t]{25ex}{$\displaystyle #1$}\parbox{5ex}{\hfil $#2$}{$\displaystyle #3$}}

\begin{itemize}
  \item [1.] Firstly, we leave to the reader to check that classical first-order models are a (trivial) special cases of the models introduced above.
        Classical relations take values in $\{0,1\}$ where $0$ is interpreted as ``true''.
        The unit ball is the whole of ${\EuScript M}$.
        The function $n_{\mbox-}$ is constantly $1$.
        To obtain the full strength of first-order logic one has to add equality as a predicate as it is absent from our logic.

  \item [2.] Secondly, any {\sc bbhu}-premodel is also a model as those in Section~\ref{uno} if we let the unit ball coincide with ${\EuScript M}$.
        The language $L_{\sf M}$ contains a relation symbol $d(x,y)$ for a metric and, possibly, symbols for all other functions and relations of ${\EuScript M}$.
        As all relations of ${\EuScript M}$ (including the metric) take values in the interval $[0,1]$, the function $n_{\mbox-}$ can be set to be the constant $1$.

  \item [3.] Now we consider an example that is non trivial and distant both from classical models and from {\sc bbhu}-models: weighted graphs.
        The set $L_{\rm fun}$ is empty and $L_{\rm rel}$ contains only a binary relation $r$.
        We require that $r^{\EuScript M}(a,b)=r^{\EuScript M}(b,a)\in[0,1]$.
        As unit ball take, again,the whole of ${\EuScript M}$.
        The function $n_{\mbox-}$ is constant $1$.


  \item [4.] Finally, and example where the unit ball is non trivial.
        Let $L_{\sf M}$ be the language of Banach spaces.
        Here we have only one symbol in $L_{\rm rel}$, the symbol $\|\mbox-\|$ for the norm.
        The set $L_{\rm fun}$ is the same as for real vector spaces in classical logic.
        Let ${\EuScript M}$ be a Banach space and define a uniformly continuous model as follows.
        The interpretation of the symbols in $L_{\rm rel}\cup L_{\rm fun}$ is the natural one.
        The unit ball of is $ M=\{a\in  M: \|a\|\le1\}$.
        

        % Let $\varphi(x)$ be an atomic formula.
        % This has form $\|t(x)\|$ for some $t(x)\in L_{\rm trm}$.  
        % Note that $\sup\{\|t(a)\|: a\in U^{|x|}\}$ does not depend on $ M$.
        % Therefore we can set it to be $n_{\varphi(x)}$

\end{itemize}

%%%%%%%%%%%%%%%%%%%%
%%%%%%%%%%%%%%%%%%%%
%%%%%%%%%%%%%%%%%%%%
%%%%%%%%%%%%%%%%%%%%
%%%%%%%%%%%%%%%%%%%%
\section{Conditions and types}

For $\varphi, \psi\in L( M)$ some closed formulas, we write \emph{${\EuScript M}\models\ \varphi\le\psi$} for $\varphi^{\EuScript M}\le\psi^{\EuScript M}$.
The meaning of \emph{${\EuScript M}\models\ \varphi=\psi$\/} and of \emph{${\EuScript M}\,\models\ \varphi<\psi$\/} is similar.

Expressions of the form $\varphi(x)\le0$ are called \emph{conditions}.
We write ${\EuScript M}\models\existsM x\,\varphi(x)\le0$ if there is an $a\in  M^{|x|}$ such that ${\EuScript M}\models\varphi(a)\le0$.
Observe that $\varphi(x)=0$ is equivalent to $|\varphi(x)|\le0$ and that $\varphi(x)\le0$ is equivalent to $\big(\varphi(x)\vee0\big)=0$.
So, when convenient, we may use equalities to denote conditions.

The negation of a condition is called a \emph{co-conditions}.
So, $\varphi(x)\neq0$, $\varphi(x)<0$ or $\varphi(x)>0$ are co-conditions.

It is easy to see that conditions are cosed under logical disjunction and logical conjunction.

\ceq{\hfill {\EuScript M}\models\varphi\vee\psi\le0}
{\Leftrightarrow}
{{\EuScript M}\models\varphi\le0 \textrm{ and }{\EuScript M}\models\psi\le0}

\ceq{\hfill {\EuScript M}\models\varphi\wedge\psi\le0}{\Leftrightarrow}{{\EuScript M}\models\varphi\le0 \textrm{ \ or \ }{\EuScript M}\models\psi\le0}

Condition are also closed under universal quanfication (over the unit ball).

\ceq{\hfill {\EuScript M}\models\bigvee_x\varphi(x)\le0}{\Leftrightarrow}{{\EuScript M}\models\forallM x\ \varphi(x)\le0}

Closure under existential quanfication is a more subtle point.
It is not true in general but it can be ensured by a small amount of saturation (cf.~Fact~\ref{fact_existential}).

A set of conditions is called a \emph{type\/} or, when $x$ is the empty tuple, a \emph{theory.}
For $p(x)\subseteq L( M)$, we define

\hfil\emph{$p(x)\le0$}\ \ =\ \ $\big\{\varphi(x)\le0 \ \  :\ \  \varphi(x)\in p\big\}.$

Up to equivalence, all types have the form above.

Beware that $p(x)$ denotes just a set of formulas, the expression $p(x)\le0$ denotes a type.

We write \emph{${\EuScript M}\models p(a)\le0$} if ${\EuScript M}\models\varphi(a)\le0$ for every $\varphi(x)\in p$.
We write \emph{${\EuScript M}\models \existsM x\ p(x)\le0$} if ${\EuScript M}\models p(a)\le0$ for some $a\in  M^{|x|}$.

%%%%%%%%%%%%%%%%%%%%%%%%%%%%%%%%%%%%
%%%%%%%%%%%%%%%%%%%%%%%%%%%%%%%%%%%%
%%%%%%%%%%%%%%%%%%%%%%%%%%%%%%%%%%%%
%%%%%%%%%%%%%%%%%%%%%%%%%%%%%%%%%%%%
%%%%%%%%%%%%%%%%%%%%%%%%%%%%%%%%%%%%
\section{Saturation}\label{saturation}

\def\ceq#1#2#3{\parbox[t]{20ex}{$\displaystyle #1$}\parbox{5ex}{\hfil $#2$}{$\displaystyle #3$}}

From this section on, we assume the existence of unboundedly many inaccessible cardinals as this simplifies the exposition.
We prove directly the existence of saturated extension (monster models), skipping the proof of the compactness theorem as this is not required in the following.
(The compactness theorem could be proved along the same lines.)

Let $p(x)\subseteq L( M)$.
We say that $p(x)\le 0$ is \emph{finitely satisfied in ${\EuScript M}$\/} if for every disjunction of formulas in $p(x)$, say $\psi(x)$, there is an $a\in  M^{|x|}$ such that ${\EuScript M}\models\psi(a)\le0$.

\begin{definition}
  We say that ${\EuScript M}$ is \emph{saturated\/} if for every $p(x)$ as in 1 and 2 below, there is an $a\in  M^{|x|}$ such that ${\EuScript M}\models p(a)\le0$
  \begin{itemize}
    \item[1.] $p(x)\subseteq L(A)$ for some $A\subseteq  M$ of cardinality $<| M|$ and $|x|=1$;
    \item[2.] $p(x)\le0$ is finitely satisfied in $ M$.
  \end{itemize}
\end{definition}

From \L\v{o}\'s Theorem we obtain that every model embeds $\mathds{I}$-elementarily in a saturated one.
First we prove the following lemma.

\begin{lemma}\label{thm_compattezza}
  Every model ${\EuScript M}$ embeds $\mathds{I}$-elementarily in a model ${\EuScript N}$ that realizes all types as in 1 and 2.
\end{lemma}

\begin{proof}
  Consider the collection of types such that 1 and 2 above.
  Assume that each type has its own set of variables and let $x$ be the concatenation of all these variables.
  We denote by $p(x)$ the union of all these types.
  Let $I$ be the set of formulas $\xi(x)$ such that $\xi(x)\le0$ is satisfied in $ M$.
  For every condition $\varphi(x)\le0$ define $X_\varphi\subseteq I$ as follows

  \ceq{\hfill X_\varphi}{=}{\Big\{\xi(x)\in I\ :\ \xi( M)\le0\ \subseteq\ \varphi( M)\le0\Big\}}

  Note that $\varphi(x)\le0$ is consistent if and only if $X_\varphi\neq\varnothing$ if and only if $\varphi(x)\in X_{\varphi}$.
  Moreover $X_{\varphi\vee\psi}\ =\ X_\varphi\cap X_\psi$. Then, as $p(x)$  is finitely consistent, the set $B=\big\{X_\varphi\,:\,\varphi(x)\in p\big\}$ has the finite intersection property.
  Extend $B$ to an ultrafilter $F$ on $I$.
  Let ${\EuScript N}$ be the ultrapower of ${\EuScript M}$ over $F$.
  That is the model with unit ball $N= M^I$ obtained as in Definition~\ref{def_ultraproduct}.

  For every formula $\xi(x)\in I$ choose some $a_\xi\in  M^{|x|}$ such that $\xi(a_\xi)\le0$.
  We may confuse $( M^I)^{|x|}$ with $( M^{|x|})^I$ as it simplifies notation.
  Let $\hat a\in ( M^{|x|})^I$ be the function that maps $\xi(x)\mapsto a_\xi$.
  By \L\v o\'s Theorem, for every formula $\varphi(x)$

  \ceq{\hfill \varphi^{\EuScript N}\big(\hat a\big)}{=}{\lim_{\xi\to F}\varphi^{ M}\big(a_\xi\big).}

  Therefore $\hat a$ realizes $p(x)\le0$ in ${\EuScript N}$.
\end{proof}

\begin{theorem}
  Every model ${\EuScript M}$ embeds $\mathds{I}$-elementarily in a saturated model.
\end{theorem}

\begin{proof}
  As usual, iterate the lemma to construct a chain of length $\lambda$, a sufficiently large inaccessible cardinal.
\end{proof}

We conclude with a convenient property of saturated models.

\begin{fact}\label{fact_existential}
  Let $ M$ is saturated.
  Then for every $\varphi(x)\in L( M)$ the following are equivalent
  \begin{itemize}
    \item[1.] $ M\models\existsM x\ \big(\varphi(x)\le0\big)$;
    \item[2.] \smash{$ M\models\displaystyle \bigwedge_x \varphi(x)\;\le\;0$}
  \end{itemize}
\end{fact}
\begin{proof}
  Only 2$\Rightarrow$1 requires a proof.
  Let $p(x)=\{\varphi(x)-\alpha: \alpha\in{\mathds R}^+\}$.
  If 2, then $p(x)\le0$ is finitely satisfied in $ M$.
  Hence 1 follows by saturation.
\end{proof}

%%%%%%%%%%%%%%%%%%%%%%%%%%%
%%%%%%%%%%%%%%%%%%%%%%%%%%%
%%%%%%%%%%%%%%%%%%%%%%%%%%%
%%%%%%%%%%%%%%%%%%%%%%%%%%%
%%%%%%%%%%%%%%%%%%%%%%%%%%%
\section{Homogeneity}

Throughout the following we fix a saturated model \emph{$\EuScript U$\/} of cardinality $\kappa$, an inaccessible cardinal larger than $|L|$, where $|L|$ stands for $\max\big\{|L_{\rm fun}|,|L_{\rm rel}|,2^\omega\big\}$.

Let $a\in{\EuScript U}^{|x|}$.
We write $p(x)={\rm tp}(a/A)$ for $p(x)=\big\{\psi(x)\in L(A):{\EuScript U}\models \psi(a)\le0\big\}$.
We write ${\rm tp}(a)$ when $A=\varnothing$.

\begin{fact}
  Let $R\subseteq{\EuScript U}^2$ be an $\mathds{I}$-relation of cardinality $<\kappa$.
  Then there is a total and surjective $\mathds{I}$-relation $S\subseteq{\EuScript U}^2$ containing $R$.
\end{fact}

\begin{proof}
  We apply the usual back-and-forth construction with a pinch of extra caution.
  Below we will make free use of Fact~\ref{fact_existential}.
  Let $a$ be an enumeration of the domain of $R$.
  Let $\bar a=\langle a_i:i<\lambda\rangle$ be an enumeration of all tuples of length $|a|$ such that $aRa_i$.
  As $\kappa$ is inaccessible, $\lambda<\kappa$.
  Let $b\in{\EuScript U}$.
  It suffices to prove that there is a $c$ such that $R\cup\{\langle b,c\rangle\}$ is an $\mathds{I}$-relation.
  Let $p(x,z)={\rm tp}(b,a/A)$ and let
  
  \ceq{\hfill q(x,\bar z)}{=}{\bigcup_{i<\lambda}p(x,z_i).}

  We claim that $q(x,\bar a)\le0$ is finitely consistent.
  A finite conjunction of formulas in $q(x,\bar a)$ has the form $\psi(x,a_{i_1})\wedge\dots\wedge\psi(x,a_{i_n})$.
  As $\psi(b,a)\le0$ and $a_{i_1},\dots,a_{i_n}\,R\;a,\dots,a$, we conclude that the condition $\psi(x,a_{i_1})\wedge\dots\wedge\psi(x,a_{i_n})\le0$ is satisfied.
  The existence of the required element $c$ follows from saturation.
\end{proof}

\begin{corollary}
  Let $a\in{\EuScript U}^{|x|}$, where $|x|<\kappa$.
  Let $A\subseteq{\EuScript U}$ have cardinality $<\kappa$.
  Then

  \ceq{\hfill p({\EuScript U})\le0}{=}{\big\{b : bRa,\ R\in{\rm Aut}({\EuScript U}/A)\big\}}
\end{corollary}


\hrulefill

Di qui in poi solo esperimenti selvaggi.


%%%%%%%%%%%%%%%%%%%%%%%%%%%%%%%%%%%
%%%%%%%%%%%%%%%%%%%%%%%%%%%%%%%%%%%
%%%%%%%%%%%%%%%%%%%%%%%%%%%%%%%%%%%
\section{Random variables}

Let ${\EuScript A}$ be a $\sigma$-algebra of subsets of $\Omega$.
Let ${\EuScript M}$ be the set of bounded $\sigma$-additive measures on ${\EuScript A}$.
Let ${\EuScript R}$ be the set of functions $f:\Omega\to{\mathds R}$ that are measurable w.r.t.\@ all measures in ${\EuScript M}$.

We define a $3$-sorted structure $\langle{\EuScript A},{}^{\circ\!}{\EuScript R},{}^{\circ\!}{\EuScript M}\rangle$.

The domain of the first sort is ${\EuScript A}$.
The language contains functions for the Boolean operations.

% On $\Omega$ we define the equivalence relation $\sim_{\EuScript A}$ as follows: $a\sim_{\EuScript A} b$ if $a\in A\leftrightarrow b\in A$ for every $A\in{\EuScript A}$.
% We make use of $\Omega$ and  $\sim_{\EuScript A}$ in the definitions of $\langle{\EuScript A},{}^{\circ\!}{\EuScript R},{}^{\circ\!}{\EuScript M}\rangle$ but they will not directly appear in the structure.

% We now define ${}^{\circ\!}{\EuScript R}$ and ${}^{\circ\!}{\EuScript M}$.
% We warn the reader that, at a deeper level these two domain contains the same elements, the difference being only in the syntax.

The domain of the second sort ${}^{\circ\!}{\EuScript R}$ contains the functions $f\in{\mathds R}$ that are linear combinations of indicator functions of sets in ${\EuScript A}$.
In the language we include the functions that make ${}^{\circ\!}{\EuScript R}$ an ${\mathds R}$-algebra (sum and and multiplications are defined pointwise).
We also include operations for the pointwise maximum and minimum of two functions.
% The unit ball, $S$, contains functions that are $\le1$ in absolute value.

The domain of the third sort, ${}^{\circ\!}{\EuScript M}$, contains signed measures $\mu:{\EuScript A}\to{\mathds R}$ that are linear combinations of measures of the form $\delta_a$, for $a\in\Omega$.
These are defined as follows: $\delta_a(A)$ is $1$ if $a\in A$ and $0$ otherwise.
The language is that of lattice vector spaces.
% The unit ball, $ M$, contains the measures in ${}^{\circ\!}{\EuScript M}$ that are $\le1$ in absolute value.

There is a function of sort ${}^{\circ\!}{\EuScript R}\times{}^{\circ\!}{\EuScript M}\to{}^{\circ\!}{\EuScript M}$ that maps $(f,\mu)$ to the measure $f\mu$ obtained multiplying in the natural way $f$ and $\mu$.

Finally, there is a predicate $I$ of sort ${\EuScript A}\times{}^{\circ\!}{\EuScript M}$ such that $I(A,\mu)=\mu(A)$.

The unit ball of $\langle{\EuScript A},{}^{\circ\!}{\EuScript R},{}^{\circ\!}{\EuScript M}\rangle$ is $\langle{\EuScript A},{}^{\circ\!\!}R, M\rangle$ where ${}^{\circ\!\!}R$ and ${}^{\circ\!\!} M$ are the set of functions, respectively measures, that are $\le1$ in absolute value.

The readers can easily convince themselfs that a bounding function exist.
Its explicit definition is not required in the sequel.

\hfil ***

The following discrete version of the Radon-Nikodym theorem is completely trivial.
For simplicity we state it for (nonnegative) measures.
As the Hahn decomposition theorem holds (trivially) in ${}^{\circ\!}{\EuScript M}$, this is no loss of generality.

\begin{fact}\label{thm_fRN}
  [Discrete Radon-Nikodym]
  Let $\mu,\nu\in{}^{\circ\!\!} M$, where ${}^{\circ\!\!} M$ is the unit ball of ${}^{\circ\!}{\EuScript M}$, be non negative.
  For every $\varepsilon > 0$ there are $E\in{\EuScript A}$ and $f\in {}^{\circ\!\!}R$, where ${}^{\circ\!\!}R$ is the unit ball of ${}^{\circ\!}{\EuScript R}$, such that
  \begin{itemize}
    \item[1.] $I(E,|\nu-\varepsilon^{-1}f\mu|)\ =\ 0$
    \item[2.] $|I(\neg E, \mu)|\le\varepsilon$.
  \end{itemize}
\end{fact}

We will use the following version of the fact above.

% \begin{fact}\label{thm_fRN2}
%   [Iterated discrete Radon-Nikodym]
%   Let $\mu,\nu\in{}^{\circ\!\!} M$, where ${}^{\circ\!\!} M$ is the unit ball of ${}^{\circ\!}{\EuScript M}$.
%   For every $n>0$ there are $E_1,\dots,E_n\in{\EuScript A}$ and $f\in{}^{\circ\!\!}R$, such that
%   \begin{itemize}
%     \item[0.] $E_n\subseteq E_{n-1}\subseteq {\dots\dots} \subseteq E_1\subseteq E_0=\Omega$
%     \item[1.] $I\big(E_i\smallsetminus E_{i+1},\ |\nu-2^{-(i+1)}f\mu|\big)\ =\ 0$
%     \item[2.] $|I(E_{i+1},\ \mu)|\le2^{-(i+1)}$;
%     \item[3.] $f_{i+1}\big[E_i\smallsetminus E_{i+1}\big]=\{0\}$.
%   \end{itemize}
% \end{fact}

% \begin{proof}
%   Apply inductively the theorem above, with $E_i$ for $\Omega$, to find $E_{i+1}$ and $f_{i+1}$ such that
%   \begin{itemize}
%     \item[1$_i$.] $I\big(E_i\smallsetminus E_{i+1},\ |\nu-2^{-(i+1)}f_{i+1}\mu|\big)\ =\ 0$;
%     \item[2$_i$.] $|I(E_{i+1},\ \mu)|\le2^{-(i+1)}$;
%     \item[3$_i$.] $f_{i+1}\big[E_i\smallsetminus E_{i+1}\big]=\{0\}$.
%   \end{itemize}
%   Finally, let $f=f_1+\dots+f_n$.
% \end{proof}


\begin{fact}\label{thm_fRN2}
  [Iterated discrete Radon-Nikodym]
  Let $\mu,\nu\in{}^{\circ\!\!} M$ be non negative.
  For every $n>0$ there are $E_1,\dots,E_n\in{\EuScript A}$ and $f\in{}^{\circ\!\!}R$ such that
  \begin{itemize}
    \item[0.] the $E_i$ are pairwise disjoint;
    \item[1.] $I\big(E_i,\ |\nu-2^if\mu|\big)\ =\ 0$
    \item[2.] $|I(X,\ \mu)|\le2^{-i}$, where $X=\neg(E_1\cup\dots\cup E_i)$;
  \end{itemize}
\end{fact}


\begin{proof}
  Apply inductively the theorem above, with $E_i$ for $\Omega$, to find $E_{i+1}$ and $f_{i+1}$ such that
  \begin{itemize}
    \item[1$_i$.] $I\big(E_i,\ |\nu-2^if_i\mu|\big)\ =\ 0$;
    \item[2$_i$.] $|I(X,\ \mu)|\le2^{-i}$, where $X=\neg(E_1\cup\dots\cup E_i)$;
    \item[3$_i$.] $f_i\big[\neg E_i]=\{0\}$.
  \end{itemize}
  Finally, let $f=f_1+\dots+f_n$.
  The third condition above (which is immediate to obtain) ensures that $f\in{}^{\circ\!\!}R$.
\end{proof}

% Write $\nu\ll\mu$ if for every $\varepsilon>0$ there is a $\delta>0$ such that $I(X,\mu)\le\delta\Rightarrow I(X,\nu)\le\varepsilon$ for all $X\in$.


% \begin{corollary}\label{cor_fRN}
%   Let $\mu,\nu\in{}^{\circ\!\!} M$ be such that $\nu\ll\mu$.
%   For every $\varepsilon > 0$ there exist an exceptional set $E\in{\EuScript A}$ and an $f\in S$, the unit ball of ${}^{\circ\!}{\EuScript R}$, such that
%   \begin{itemize}
%     \item[1.] $I(X,1,\nu)\ =\ I(X,f/\varepsilon,\mu)$ for evey $X\subseteq\neg E$;
%     \item[2.] $\displaystyle\sup_{X\subseteq E}|I(X, \mu)|\le\varepsilon$.
%   \end{itemize}
% \end{corollary}

% We note that 1 \& 2 of Theorem~\ref{thm_fRN} can be expressed by a condition.
% Infact, if $\varphi(X)$ is a formula, where $X$ is a variable of sort ${\EuScript A}$, then $\sup_{Y\subseteq X}\varphi(Y)$ is a formula --~as it is equivalent to $\sup_Y\varphi(Y\cap X)$.
% Also, recall that conditions are closed under the Boolean operations of conjunction and disjunction.
% They are also closed under universal quantification.

We now consider a saturated extension of $\langle{\EuScript A},{}^{\circ\!}{\EuScript R},{}^{\circ\!}{\EuScript M}\rangle$ that we denote by $\langle{}^{*\!\!\!}{\EuScript A},{}^{*\!}{\EuScript R},{}^{*\!}{\EuScript M}\rangle$.
The following fact is a direct consequence of saturation.

\begin{fact}
  Let $\mu\in{\EuScript M}$.
  Then there is ${}^{*\!\!}\mu\in{}^{*\!}{\EuScript M}$ such that $\mu_{\restriction{\EuScript A}}={}^{*\!\!}\mu_{\restriction{\EuScript A}}$.
\end{fact}

\begin{proof}
  It suffices to check that the following type is finitely consistent in ${\EuScript A}$ (read $\nu$ as a variable and $\mu(X)$ as a real number)
  $$
  \big\{\nu(X)=\mu(X) \ :\ X\in{\EuScript A}\big\},
  $$
  which immediate.
\end{proof}

We can improve on the fact above.
In fact, we can also control the value of ${}^{*\!\!}\mu(A)$ for any $A\in{\EuScript A}$.

\def\ceq#1#2#3{\parbox[t]{25ex}{$\displaystyle #1$}\parbox{5ex}{\hfil $#2$}{$\displaystyle #3$}}

\begin{fact}
  Let $\mu$ be a bounded measure on ${\EuScript A}$, not necessarily in ${}^{\circ\!}{\EuScript M}$.
  Let $A\in{}^{*\!\!\!}{\EuScript A}$.
  Define 
  
  \ceq{\hfill m_{\rm in\,}}{=}{\sup\{\mu(X): X\in{\EuScript A},\ X\subseteq A\}}

  \ceq{\hfill m_{\rm ex}}{=}{\,\inf\{\mu(X): X\in{\EuScript A}, \  A\subseteq X \}}

  Then for every $m_{\rm in\,}\le r\le m_{\rm ex}$ there is a measure ${}^{*\!\!}\mu\in{}^{*\!}{\EuScript M}$ such that ${}^{*\!\!}\mu_{\restriction{\EuScript A}}=\mu_{\restriction{\EuScript A}}$ and  ${}^{*\!\!}\mu(A)=r$.
\end{fact}

\begin{proof}[Proof (sketch)]
Assume for simplicity that $\mu$ is bounded by $1$.
It suffices to check that the following type is finitely consistent (read $\nu$ as a variable and $\mu(X)$ as a real number).
$$
\big\{\nu(X)=\mu(X) \ :\ X\in{\EuScript A}\big\}
\ \ \cup\ \ \
\big\{\nu(A)=r\big\}.
$$
In fact, any ${}^{*\!\!}\mu$ realizing this type is as required by the lemma.

Let ${\EuScript A}'\subseteq{\EuScript A}$ be a finite Boolean algebra.
We define a measure $\nu$ on the Boolean algebra generated by $A,X_1,\dots,X_n$ that satisfies the type above restrected to ${\EuScript A}'\cup\{A\}$.
Let $X_1,\dots,X_n$ be the atoms of ${\EuScript A}'$.
Assume $A\notin{\EuScript A}$, to avoid trivialities.
Then there are some sets $X_i$ such that both $X_i\cap A$ and $X_i\smallsetminus A$ are nonempty.
Suppose these sets are $X_1,\dots,X_m$.
For $i\le m$ we define $\nu(X_i\cap A)=\varepsilon\,\mu(X)$ and $\nu(X_i\smallsetminus A)=(1-\varepsilon)\mu(X_i)$, where $0\le\varepsilon\le1$ is specified below.
For $i>m$ let $\nu(X_i)=\mu(X_i)$.
Note that $\mu(X_1\cup\dots\cup X_m)\ge m_{\rm ex}-m_{\rm in}$.
Therefore with a suitable $\varepsilon$, we can obtain $\nu(A)=r$.
\end{proof}

If ${\EuScript A}$ is a $\sigma$-algebra and $\mu$ is a $\sigma$-additive measure then the supremum and the infimum in the fact above are attained.
We will use the following.

\begin{corollary}
  Let $\mu\in{\EuScript M}$.
  Then for every $A\in{}^{*\!\!\!}{\EuScript A}$ there are $A_{\rm in},A_{\rm ex}\in{\EuScript A}$ such that $A_{\rm in}\subseteq A\subseteq A_{\rm ex}$ and a measures ${}^{*\!\!}\mu$ such that 
  ${}^{*\!\!}\mu_{\restriction{\EuScript A}}=\mu_{\restriction{\EuScript A}}$ and ${}^{*\!\!}\mu(A)=\mu(A_{\rm ex})$.  
  A similar claim holds for $A_{\rm in}$.
  %We can also require it holds simultaneously for any family of sets $\{A_i:i<\lambda\}\subseteq{}^{*\!\!\!}{\EuScript A}$ of small cardinality (???).
\end{corollary}
\end{document}

\begin{question}
  Let $\mu\in{\EuScript M}$.
  Let ${}^{*\!\!}\mu\in{}^{*\!}{\EuScript M}$ be as above.
  Let $f\in{}^{*\!\!}R$, the unit ball of ${}^{*\!}{\EuScript R}$, be given and assume $f$ is non negative.
  We define the function ${}^{\mu\!\!}f:\Omega\to{\mathds R}$ as follows (tentative)
  $$
  {}^{\mu\!\!}f(a)\ =\ \inf\big\{\alpha : I(A, f\,{}^{*\!\!}\mu)\le \alpha \mu(A) \ :\ A\in{\EuScript A},\ a\in A\big\}
  $$
  Is it true that for every $A\in{\EuScript A}$ 
  $$
  \int_A{}^{\mu\!\!}f{\rm d}\mu\ =\ I(A,f\,{}^{*\!\!}\mu)\ \ ?
  $$
\end{question}

\begin{theorem}[Radon-Nikodym] 
  Let $\nu,\mu\in{\EuScript M}$ be such that $\nu\ll\mu$.
  Then there is an $f\in{\EuScript R}$ such that 
  $$
  \int_X f{\rm d}\mu = \int_X {\rm d}\nu
  $$
  for every $X\in{\EuScript A}$.
\end{theorem}

\begin{proof}
  ???????????????????
\end{proof}


%%%%%%%%%%%%%%%%%%%%%%%%%%%%%%%%%%%
%%%%%%%%%%%%%%%%%%%%%%%%%%%%%%%%%%%
%%%%%%%%%%%%%%%%%%%%%%%%%%%%%%%%%%%
% \section{Completeness vs.\@ saturation\quad !` please expand !}

% The $A$-limit uniformity on $ M^{|x|}$ is the uniformity that has the following entourages 

% \ceq{\hfill V_{\varphi(x),\, \varepsilon}}{=}{\Big\{(a,b)\in  M^{|x|}\times  M^{|x|}\quad :\quad \big|\varphi^{\EuScript M}(a)-\varphi^{\EuScript M}(b)\big|<\varepsilon\Big\}}

% for $\varphi(x)\in L(A)$ and $\varepsilon\in{\mathds R}^+$. It is the coarsest uniformity that makes all formulas in $L(A)$ uniformly continuous.

% Let $p(x)\subseteq L( M)$ be a type.
% We say that $p(x)\le0$ is \emph{Cauchy\/} (in the $A$-limit uniformity) if for every $\varphi(x)\in L(A)$ and every $\varepsilon\in{\mathds R}^+$ there is a disjunction of formulas in $p(x)$, say $\psi(x)$, such that $\varnothing\neq\big(\psi( M)\le0\big)^2\subseteq V_{\varphi(x),\, \varepsilon}$.

% \begin{fact}
%   Let $\lambda=L(A)$.
%   If ${}^{\circ\!\!} M$ is $\lambda$-saturated then and $p(x)\subseteq L( M)$ is Cauchy, then $p(x)$ is realized in ${}^{\circ\!\!} M$.
% \end{fact}


% For every $\varphi(x)\in\ L(A)$ and $n\in{\mathds N}$ let 


% Suppose ${\EuScript N}\models p(a)\le0$ and let $\big(\psi( M)\le0\big)^2\subseteq V_{\varphi(x),\, \varepsilon}$


\end{document}
